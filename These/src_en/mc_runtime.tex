\subsection{Runtime}
A runtime is a piece of software used by others software to abstract part of the system.
%
The main idea behind the runtime is {\em build once, run anywhere}.
%
There are several type of runtime.
%
Some high level programming languages use runtime, for example Java has a runtime for garbage collection.
%
Also some implementations of MPI have a runtime.
%
Task based programming also tend to have a runtime.
%
In the following part, we will focus on runtime support for task based programming.



The main interest of runtime in task based programming is to schedule tasks while respecting strict dependencies order.
%Le rôle des supports d'exécution pour la programmation en tâches est d'orchestrer la réalisation des tâches en respectant leurs dépendances.
%
These runtimes also provide a load balancing over all available hardware ressources (potentially heterogeneous) while keeping data consistency.
%Ils se chargent de la répartition des tâches sur les ressources matérielles disponibles (potentiellement hétérogènes) tout en assurant la cohérence des données.
%
Some runtime also provide memory transfer between ressources, for example between main memory and a graphical card or just between two processes.
%
These transfers could be implicit, the runtime has data knowledge, or they could be explicit, a special task.


Some runtime use schedule policy (static or dynamic) to improve load balancing.
%
But the perfect scheduler doesn't exist maybe will never exist.
%
Indeed, finding the best schedule for a set of tasks on a limited number of resources machine is a NP-complete problem\footnote{A problem is NP-complete if the solving time is polynomial compare to input data.}.
%
Some heuristic are used to obtain some reasonably good results, for example the greedy algorithm.
%
Also, some heuristic needs more information about tasks, like an estimated cost.
