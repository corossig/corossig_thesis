\subsection{Simulation of a simple physical example}
Let's take a simple physical example to see how linear algebra can be used in physical simulation.
%
We want to simulate the pressure of an oil column.
%
We know that the density of the oil presents in the column is $\rho = 0.9192~kg/m^3$.
%
We also know the physical equation of the hydrostatic pressure:
%
\begin{equation}
\label{eq:hydrostatic}
\frac{\mathrm d P}{\mathrm d z} = \rho{}g
\end{equation}
%
where $P$ is the pressure, $z$ is the depth and $g$ is the gravitational acceleration.
%
By using the Taylor's theorem at first order, we obtain the equation:
%
\begin{equation}
P(z_0+h) = P(z_0) + h \frac{\mathrm d P}{\mathrm d z} (z_0) + o(h^2)
\end{equation}
\begin{equation}
\frac{\mathrm d P}{\mathrm d z} (z_0) = \frac{P(z_0+h) - P(z_0)}{h} + o(h^2)
\end{equation}

%   (-_-)   %
\begin{figure}[!ht]
  \centering
  \includegraphics[width=0.5\textwidth]{rocks}
  \caption{oil column scheme.}
  \label{fig:oil_schema}
\end{figure}
%
We discretize the problem in $n$ cells with a finite difference method (fig.~\ref{fig:oil_schema}): let us consider $Z_i$ the approximation of $z$ on the cell $i$, $i$ going from $0$ to $n-1$.
%
Each cells are separate by a distance $h$ called $\Delta{z}$:
%
\begin{equation}
\label{eq:taylor_fd}
\frac{\mathrm d P}{\mathrm d z}(Z_i) \approx \frac{P(Z_{i}) - P(Z_{i-1})}{\Delta{z}}
\end{equation}
%
By injecting \eqref{eq:taylor_fd} in \eqref{eq:hydrostatic} leads to:
%
\begin{equation}
\frac{P(Z_{i}) - P(Z_{i-1})}{\Delta{z}} = \rho{}g
\end{equation}
\begin{equation}
\label{eq:system_pressure}
P(Z_{i}) - P(Z_{i-1}) = \rho{}g\Delta{z}
\end{equation}
We also have the boundary condition that at ground level 0, the pressure is 1000~hPa or $10^5$~Pa:
%
\begin{equation}
P(Z_0) = 10^5
\end{equation}
%
We can now write the entire system under a matrix form with $n$ cells:
%
\begin{equation}
\label{eq:ax_b}
\begin{bmatrix}
   1   &    0   &    0   & \cdots & \cdots & \cdots & \cdots &   0    \\
  -1   &    1   &    0   & \ddots &        &        &        & \vdots \\
   0   &   -1   &    1   &    0   & \ddots &        &        & \vdots \\
\vdots & \ddots & \ddots & \ddots & \ddots & \ddots &        & \vdots \\
\vdots &        & \ddots & \ddots & \ddots & \ddots & \ddots & \vdots \\
\vdots &        &        & \ddots &   -1   &    1   &    0   &   0    \\
\vdots &        &        &        & \ddots &   -1   &    1   &   0    \\
   0   & \cdots & \cdots & \cdots & \cdots &    0   &   -1   &   1    \\
\end{bmatrix}
\begin{pmatrix}
  P(Z_0)  \\
  P(Z_1)  \\
\vdots \\
\vdots \\
\vdots \\
\vdots \\
P(Z_{n-2}) \\
  P(Z_{n-1})  \\
\end{pmatrix}
=
\begin{pmatrix}
 10^5  \\
\rho{}g\Delta{z}     \\
\vdots \\
\vdots \\
\vdots \\
\vdots \\
\rho{}g\Delta{z} \\
\rho{}g\Delta{z}    \\
\end{pmatrix}
\end{equation}
By doing the multiplication of each line of $A$ by $x$, we obtain exactly the system of equations \eqref{eq:system_pressure}.
%
Now, we have a matrix $A$ multiply a vector $x$ equal a vector $b$, it's time to talk about linear algebra.
