\subsection{Décomposition de domaine}
Lorsque le problème à résoudre devient trop gros pour être traité sur une seule machine, il est nécessaire d'adapter la méthode de résolution du problème.
Pour la simulation de réservoir, les coefficients de la matrice représentent les interactions entre les cellules du réservoir.
%
Donc si nous partitionnons le graphe de connexions entre les cellules, chaque unité de calcul sera en charge d'un ensemble de cellules (Fig.~\ref{fig:domain}).
%
Nous obtenons donc un préconditionneur totalement parallèle et un bon équilibrage de charge aussi bien en terme de volume de donnée que de volume de calcul.
%
Les autres opérations du GMRES peuvent aussi être faites en parallèle moyennant des opérations de synchronisations coûteuses telles que des réductions à la fin de chaque opération.


En utilisant la méthode de Schwarz additive, ce préconditionneur est totalement parallèle et il ignore les interactions inter-domaines.
%
Malheureusement, plus il y a d'interactions ignorées, moins le préconditionneur est efficace.
%
Formulé différemment, le nombre de domaines aura un impact la convergence du GMRES.
