\subsubsection{Mémoire distribuée}
En mémoire distribuée, nous n'effectuons pas les mêmes calculs qu'en mémoire partagée, mais les accélérations obtenues nous donnerons une approximation des performances que nous pourrons obtenir.
%
Sur la machine Rostand, la factorisation atteint une accélération de 11,8 et la résolution triangulaire une accélération de 3,5.
%
Sur la machine Manumanu, cette accélération monte à 93 pour la factorisation (Fig.~\ref{fig:res_facto_mpi_manu}) et à 73 pour la résolution triangulaire (Fig.~\ref{fig:res_trsv_mpi_manu}).



\subsubsection{First touch}
Les résultats de la factorisation et de la résolution triangulaire avec une allocation first touch sur la machine  sont exposés dans le chapitre précèdent.
%
Les résultats ne sont pas aussi bons que ceux que nous pourrions obtenir avec une meilleure gestion de la mémoire.
%
Pour rappel, nous obtenions au mieux une accélération de 8,7 sur coeurs pour la factorisation et une accélération de 2,8 pour la résolution triangulaire.


Sur la machine Manumanu, nous obtenons le même type d'accélération que pour le SpMV.
%
Tant que nous utilisons moins de 2 bancs NUMA, nous obtenons une accélération de 10 pour la factorisation (Fig.~\ref{fig:res_facto_ft_manu}) et une accélération de 6,2 pour la résolution triangulaire (Fig.~\ref{fig:res_trsv_ft_manu}).


\subsubsection{Interleave}
Pour essayer de diminuer les effets NUMA, nous activons la politique d'allocation mémoire interleave.
%
Les pages mémoires sont donc distribuées uniformément entre chaque banc NUMA.
%
Sur Rostand, la factorisation donne des résultats légèrement moins bons qu'avec une politique d'allocation first touch (Fig.~\ref{fig:res_facto_inter_rostand})
%
Par contre, nous obtenons une amélioration entre 3~\% et 30~\% de la résolution triangulaire pour un nombre faible de variables primaires (Fig.~\ref{fig:res_trsv_inter_rostand}).



Sur Manumanu, la factorisation se comporte de la même façon que sur Rostand (Fig.~\ref{fig:res_facto_inter_manu}).
%
De même, l'accélération maximale de la résolution triangulaire est meilleure avec une politique d'allocation interleave (Fig.~\ref{fig:res_trsv_inter_manu}).




\subsubsection{NATaS}
\`A la différence des autres ordonnanceurs, NATaS va tenir compte de l'affinité NUMA des tâches.
%
Cette affinité a été définit par Taggre de tel sorte à équilibrer la charge sur les différents bancs NUMA.
%
NATaS offre de meilleurs performances sur Rostand par rapport à la politique d'allocation interleave.
%
La factorisation est 40\% plus rapide avec 8 variables primaires et la résolution triangulaire est 23\% plus rapide.
%
Avec 1 variable primaire nous n'obtenons pas de gain sur la résolution triangulaire.
