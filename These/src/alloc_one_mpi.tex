\subsection{Un processus MPI par banc NUMA}
Les problèmes rencontrés sur les machines NUMA proviennent essentiellement de la mémoire partagée.
%
Lorsque deux threads partagent un espace mémoire et que ces deux threads s'exécutent sur des bancs NUMA différents, il y aura potentiellement des accès mémoire non performants.
%
Nous pouvons donc résoudre le problème des effets NUMA en utilisant plusieurs processus.
%
En utilisant un processus MPI par banc NUMA et une politique d'allocation first touch ou bind, on se retrouve toujours avec des accès mémoire du banc mémoire le plus proche.
%
Mais la problématique reste similaire à la problématique du choix entre la parallélisation en mémoire distribuée et la parallélisation en mémoire partagée, ce n'est pas toujours possible ou performant.
%
Dans le cas où les algorithmes en mémoire distribuée ne passeraient pas à l'échelle, il est nécessaire de noter que l'utilisation de cette solution multipliera le nombre de processus MPI par le nombre de bancs NUMA.
%
Notre application utilise déjà du parallélisme en mémoire distribuée, il n'y aura donc pas de modification à effectuer pour cette solution.
%
Par contre, les algorithmes que nous utilisons donnent de meilleurs résultats avec peu de processus MPI, nous cherchons donc à limiter le nombre de processus MPI.
