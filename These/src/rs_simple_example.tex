\subsection{Simulation of a simple physical example}
Let's take a simple physical example to see how linear algebra is used in physical simulation.
%
We want to simulate the pressure of an oil column.
%
The density of the oil is $\rho = 0.9192~kg/m^3$.
%
We use the physical equation of the hydrostatic pressure :
%
\begin{equation}
\frac{\mathrm d P}{\mathrm d z} = \rho{}g
\end{equation}
%
where $P$ is the pressure, $z$ is the depth and $g$ is the gravitational acceleration.
%
With the Taylor's theorem at first order :
%
\begin{equation}
P(x_0+h) = P(x_0) + h P'(x_0) + o(h)
\end{equation}
\begin{equation}
P'(x_0) = \frac{P(x_0+h) - P(x_0)}{h} + o(h)
\end{equation}
%
We discretize the problem with a finite difference method, each element are separate by a $\Delta{z}$ distance :
%
\begin{equation}
P'(X_i) \approx \frac{P(X_{i}) - P(X_{i-1})}{\Delta{z}}
\end{equation}
%
By using a finite element method on the function $P$ :
%
\begin{equation}
\frac{P(X_{i}) - P(X_{i-1})}{\Delta{z}} \approx \rho{}g
\end{equation}
\begin{equation}
\label{eq:system_pressure}
P(X_{i}) - P(X_{i-1}) \approx \rho{}g\Delta{z}
\end{equation}
We also have the boundary condition that at ground level 0, the pressure is 1000~hPa or $10^5$~Pa:
%
\begin{equation}
P(X_0) = 10^5
\end{equation}
%
We can now write the entire system under a matrix form with $n$ cells :
%
\begin{equation}
\label{eq:ax_b}
\begin{bmatrix}
   1   &    0   &    0   & \cdots & \cdots & \cdots & \cdots &   0    \\
  -1   &    1   &    0   & \ddots &        &        &        & \vdots \\
   0   &   -1   &    1   &    0   & \ddots &        &        & \vdots \\
\vdots & \ddots & \ddots & \ddots & \ddots & \ddots &        & \vdots \\
\vdots &        & \ddots & \ddots & \ddots & \ddots & \ddots & \vdots \\
\vdots &        &        & \ddots &   -1   &    1   &    0   &   0    \\
\vdots &        &        &        & \ddots &   -1   &    1   &   0    \\
   0   & \cdots & \cdots & \cdots & \cdots &    0   &   -1   &   1    \\
\end{bmatrix}
*
\begin{pmatrix}
  P(X_0)  \\
  P(X_1)  \\
\vdots \\
\vdots \\
\vdots \\
\vdots \\
P(X_{n-1}) \\
  P(X_n)  \\
\end{pmatrix}
=
\begin{pmatrix}
 10^5  \\
\rho{}g\Delta{z}     \\
\vdots \\
\vdots \\
\vdots \\
\vdots \\
\rho{}g\Delta{z} \\
\rho{}g\Delta{z}    \\
\end{pmatrix}
\end{equation}
By doing the multiplication of each line of $A$ by $x$, we obtain exactly the system of equations \ref{eq:system_pressure}.
%
Now, we have a matrix $A$ multiply a vector $x$ equal a vector $b$, it's time to talk about linear algebra.
