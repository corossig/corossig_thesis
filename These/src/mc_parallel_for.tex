\subsection{Parallélisme de boucle}
Le parallélisme de boucle est un paradigme qui peut être utilisé sur des machines à mémoire partagée.
%
Il s'agit de traiter en parallèle toutes les itérations d'une boucle en les distribuant équitablement sur tous les coeurs de calcul disponibles.
%
Il faut bien sûr que ces itérations soit indépendantes, c'est à dire qu'avec un nombre infini de coeur de calcul, on pourrait traiter une itération par coeur simultanément.
%
Ce paradigme fonctionne de la manière suivante : un thread est crée par coeur de calcul et chaque thread doit s'occuper de traiter une partie des itérations de la boucle.
%


L'interface de programmation qui a le plus démocratisé ce paradigme est OpenMP.
%
Cette interface utilise les directives de compilation en C qui ont l'avantage d'être simple à utiliser et elles peuvent se désactiver facilement pour retrouver un code séquentiel.
%
En ajoutant la fameuse directive ``\#pragma omp parallel for'' juste au-dessus d'une boucle for, on obtient facilement un programme multi-threadé, la description du parallélisme est très simple.
%
Les performances obtenues avec ce paradigme sont souvent suffisantes pour un grand nombre de logiciel et le ratio entre le temps de développement et le gain en temps d'exécution est imbattable.
%
Malheureusement, il peut arriver qu'il y ait des dépendances de données entre deux itérations, dans ce cas, ce paradigme de parallélisation donnera un résultat faux et il faudra utiliser un autre paradigme.
