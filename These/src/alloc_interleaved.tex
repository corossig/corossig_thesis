\subsection{Interleaved memory}
First touch n'étant pas parfait, il est nécessaire d'avoir d'autres politiques d'allocation.
%
La politique {\em Interleaved memory} distribue uniformément les pages mémoires sur tous les bancs mémoires en mode tourniquet.
%
Cette distribution est faite par le noyau du système d'exploitation au moment où une page mémoire est utilisée pour la première fois par le programme.
%
Sur un système d'exploitation utilisant Linux comme noyau, il suffit d'utiliser la commande {\em numactl --interleave=all ./programme} pour utiliser cette politique d'allocation dans tout le programme.
%
En plus d'avoir très peu d'impact sur le code source d'une application, la politique d'entrelacement mémoire montre des atténuations des effets NUMA dans le cas général.
%
En moyenne, il n'y a pas d'amélioration de la latence, mais la bande passante est améliorée grâce à l'utilisation simultanée de tous les liens mémoires par rapport à une initialisation séquentielle avec une politique first touch.
%
Ainsi, il est généralement intéressant d'expérimenter cette politique, avant d'étudier la question des optimisations NUMA.
%
Dans notre cas, cette politique nous donnait de meilleurs performances que la politique first touch.
%
Mais les résultats n'étaient pas suffisant.
