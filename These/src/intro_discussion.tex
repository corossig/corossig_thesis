La simulation numérique est une opération essentielle dans beaucoup d'entreprise.
%
Ce type de simulation se base des modèles physiques représentés sous la forme d'équation.
%
L'algèbre linéaire permet de résoudre ces équations.
%
Mais ces résolutions sont coûteuses en temps de calcul.
%
C'est pourquoi certaines techniques comme les méthodes itératives ont été inventées.
%
Ces méthodes fournissent rapidement une solution acceptable à nos problèmes.


Pour obtenir de plus en plus de puissance de calcul, l'architecture des ordinateurs s'est complexifiée.
%
Ainsi de nombreux paradigmes de parallélisation sont apparut.
%
Parmi ces paradigmes, la programmation à base de graphe de tâches offre le plus de flexibilité.



Les efforts entrepris pour la programmation hybride CPU/CPU ont permit l'émergence de nouveaux runtimes.
%
Ces runtimes prennent en considération de nouveaux paramètres tel que le poids des tâches de calcul.
%
Ce type de parallélisation est souvent utilisé en algèbre linéaire dense.
%
Les méthodes de parallélisation actuels peuvent aussi s'appliquer à des noyaux d'algèbre linaire creuse mais la granularité des tâches de calcul ne permet d'obtenir une parallélisation efficace.




Toutes ces solutions donc aident le programmeur à écrire un programme parallèle efficace.
%
Mais il est toujours du devoir du programmeur de choisir un grain de calcul adapté au type d'ordonnancement choisi.
%
Dans le chapitre suivant, nous allons nous consacrer à étudier ce problème et nous allons essayer de trouver une réponse qui satisfasse au moins les problèmes rencontrés en algèbre linéaire creuse.



