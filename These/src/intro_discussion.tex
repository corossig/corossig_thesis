Les méthodes de parallélisation actuels peuvent s'appliquer à des noyaux d'algèbre linaire creuse.
%






Il existe plusieurs formes de parallélisme et diverses manières de les exploiter.
%
Les solutions
%
La composabilité des diverses solutions peut aussi poser problème.
%
Par exemple une solution comme PaRSEC permet d'exprimer du parallélisme de tâche en mémoire distribuée mais ne permet pas d'exprimer facilement du parallélisme de boucle.
%
Dans cet exemple, nous pouvons vouloir utiliser en plus de PaRSEC un autre runtime qui s'occupera du parallélisme de boucle.
%



Lorsque nous souhaitons paralléliser



Les efforts entrepris pour la programmation hybride CPU/CPU ont permit l'émergence de nouveaux runtimes.
%
Ces runtimes prennent en considération de nouveaux paramètres tel que le poids des tâches de calcul.


Toutes ces solutions aident le programmeur à écrire un programme parallèle efficace.
%
Mais il est toujours du devoir du programmeur de choisir un grain de calcul adapté au type d'ordonnancement choisi.
%
Dans le chapitre suivant, nous allons nous consacrer à étudier ce problème et nous allons essayer de trouver une réponse qui satisfasse au moins les problèmes rencontrés en algèbre linéaire creuse.




Algèbre linéaire dense utilise déjà des runtimes.
