\subsection{Processeurs multi-coeur}
Pour obtenir encore plus de parallélisme, il est possible de multiplier les unités de calcul au sein d'un processeur.
%
Ces unités de calcul, aussi appelées coeurs de calcul, peuvent être considérées comme des processeurs.
%
Chaque coeur a son propre pipeline d'instructions, ses registres et ses unités arithmétiques.
%
Les coeurs partagent un ou plusieurs niveaux de cache entre eux ainsi que le bus d'accès à la mémoire.
%
La puissance de calcul d'un processeur à 4 coeurs composés d'unités vectoriels et chaque coeur tournant à 4~GHz est de 64~GFLOPS.
%
Encore une fois, cette puissance de calcul est théorique, il faut que le programme puisse utiliser tous les coeurs du processeur ainsi que les instructions vectorielles.
%
Pour pouvoir utiliser tous les coeurs, il faut utiliser du parallélisme multi-coeur.
