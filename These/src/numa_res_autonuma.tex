\subsection{\'Equilibrage automatique NUMA}
Les noyaux Linux récents proposent un équilibrage de charge automatique des pages mémoires.
%
Malheureusement, nous ne pouvons pas utiliser les grappes de serveurs à notre disposition pour tester cette fonctionnalité.
%
La version de Linux disponible sur ces machines n'est pas assez récente, la fonctionnalité autoNUMA n'est apparue que dans la version 3.13 du noyau.
%
\`A la place, nous allons utiliser une machine de bureau contenant deux processeurs Intel Xeon X5660, chaque banc NUMA dispose de 6 coeurs de calculs et de 24~Go de mémoire vive.
%
La version de Linux utilisée est la 3.18.
%
Cette méthode ne fonctionne que lorsque le programme est exécuter suffisamment longtemps pour avoir le temps d'analyser toute la mémoire utilisée.
%
Nous allons donc faire varier le nombre de résolutions GMRES effectuées pour savoir à partir de quand cette méthode devient intéressante.
