\subsection{Les opérateurs d'agrégations}
Nous appelons {\em opérateurs d'agrégations} les différentes heuristiques utilisées pas Taggre pour grossir un graphe de tâches.
%
Ces heuristiques ont pour règle de garder la propriété acyclique du graphe de tâches.
%
Quatre heuristiques ont été créées, chaque opérateur s'occupe de résoudre un problème spécifique et aucun d'entre eux ne peut créer de cycle.
%
La création d'un cycle lors d'une agrégation intervient lorsque l'on crée un groupe de tâche dans lequel deux des tâches ont une dépendance indirecte, symbolisé par un chemin dans le graphe, et qu'au moins une des tâches de ce chemin n'appartient pas au groupe.



On pourrait se poser la question de l'utilisation d'un partitionneur de graphe comme opérateur d'agrégation.
%
En effet, les partitionneurs de graphe essaient de créer des groupes de noeud proche spatialement.
%
Si nous prenons en compte ce seul paramètre, ils feraient des opérateurs de très bonne qualité.
%
Malheureusement, les partitionneurs ne travaillent que sur des graphes non orientés.
%
Le résultat de ces opérateurs serait donc inutilisable parce que les graphes quotients obtenus faire apparaître des cycles.
