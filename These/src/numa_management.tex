De manière générale, quand un programme alloue de la mémoire, il reçoit un pointeur sur de la mémoire virtuelle.
%
Cette mémoire virtuelle est unique à chaque processus et est partagée entre les threads.
%
La relation entre la mémoire virtuelle et la mémoire physique est faite par le système d'exploitation.
%
Il est en charge de la bonne gestion de la mémoire physique.
%
Lorsqu'un processus accède à de la mémoire virtuelle qui n'as pas encore été mappée à de la mémoire physique, une faute de page est générée, on appelle ça toucher une page.
%
Cette faute est traitée par le système d'exploitation, il s'occupera de trouver une page mémoire libre dans la mémoire physique et il la fera correspondre à une adresse virtuelle.
%
Lors des accès mémoire du processus, la translation de l'adresse virtuelle vers l'adresse physique est faite par un des composant du processeur, l'unité de gestion mémoire ou MMU\footnote{Memory Management Unit}.
%
De cette façon, le processus ne voit que l'espace d'adressage virtuelle et ne connaît rien de l'espace d'adressage physique.
%
Le système d'exploitation peut donc changer l'emplacement physique d'une page sans affecter le fonctionnement du processus, c'est totalement transparent.
%
Toutefois, l'emplacement physique d'une page virtuelle peut impacter les performance du processus sur les architectures NUMA.
%
Cet impact dépendra des connexions entre le banc mémoire physique où se situe la page et le coeur de calcul qui fait tourner le processus.


Sur une machine NUMA, lorsqu'une faute de page arrive, le système d'exploitation doit choisir sur quel banc NUMA il placera la page.
%
Avec Linux, il y a au moins trois politiques d'allocations de disponibles :
\begin{itemize}
        \item {\em First Touch}: La mémoire est allouée sur le banc NUMA du coeur de calcul qui y accède en premier.
                         Il s'agit de la politique par défaut.
        \item {\em Bind}: La mémoire est allouée sur banc NUMA spécifié en paramètre.
        \item {\em Interleaved}: Les allocations mémoire sont entrelacées parmi tous les bancs NUMA disponibles.
\end{itemize}
Sur Linux, ces politiques peuvent être choisies avec l'appel système {\em mbind}, ou avec l'outil en ligne de commande {\em numactl}.
%
La version 3.13 de Linux apporte un nouveau mécanisme de gestion de la mémoire sur les machine NUMA, il s'agit d'AutoNUMA.
%
Ce mécanisme a pour but d'optimiser le placement des pages NUMA tout long de l'exécution d'un processus (Voir section~\ref{sec:autonuma}).


D'autres système d'exploitation peuvent avoir leurs propres ensembles de politiques d'allocations NUMA.
%
Solaris, par exemple, fournit aussi la politique {\em next-touch}~\cite{next_touch}.
%
Avec cette politique, les pages mémoires physique seront déplacées près du prochain coeur de calcul qui y accédera (Voir section~\ref{sec:next_touch}).
