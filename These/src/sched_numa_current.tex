\subsection{Statuts des ordonnanceurs actuels}
L'ordonnanceur de tâches a suffisamment d'information pour avoir une bonne gestion de la mémoire.
%
Certains ordonnanceurs ont un contrôle total de la mémoire utilisée par les tâches,




Parsec de son côté est l'un des seuls ordonnanceurs à offrir un support des architectures NUMA.
%
Par contre son support ressemble plus à la solution mémoire distribuée.
%
En effet, le support du NUMA est fait avec les structures Virtual Process (vp) de PaRSEC, ce qui peut correspondre à avoir un processus MPI par banc NUMA.
%
Mais ce n'est pas si grave, le vol de tâche entre vp existe.
%
Il conserve donc l'aspect équilibrage de charge des solutions multithreadées.
%
Par contre, cette solution ne convient toujours pas à résoudre notre problème, nous essayons d'avoir le moins possible de parallélisme en mémoire distribuée.


Il existe aussi des tentatives d'ajout du support de la localité des données à OpenMP (ForestGOMP~\cite{Bro10Thesis}).
