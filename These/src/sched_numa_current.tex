\subsection{Statuts des ordonnanceurs actuels}
La gestion du placement des pages mémoires n'est pas utile si le code qui utilisera ces données ne s'exécute pas sur le bon banc NUMA.
%
Dans le cas de la programmation par tâche, chaque tâche doit connaître le banc NUMA qui lui est le plus favorable.
%
Cette information pourra être ensuite donnée à l'ordonnanceur de tâches qui s'occupera de placer correctement la tâche.
%
Actuellement, certains ordonnanceurs ont un contrôle total de la mémoire utilisée par les tâches, ils pourraient donc optimiser l'affinité NUMA des programmes, mais ils sont très peu à le faire.


PaRSEC est un cadriciel de parallélisation par tâche qui fonctionne aussi en mémoire distribuée.
%
Il est l'un des seuls ordonnanceurs à offrir un réel support des architectures NUMA.
%
Par contre son support est une analogie avec la programmation en mémoire distribuée.
%
En effet, le support du NUMA est fait avec les structures Virtual Process (VP) de PaRSEC, ce qui peut correspondre à avoir un processus MPI par banc NUMA.
%
Mais ce n'est pas si grave, le vol de tâche entre VP existe.
%
Il conserve donc l'aspect équilibrage de charge des solutions multithreadées.
%
Par contre, cette solution ne convient toujours pas à résoudre notre problème, nous essayons d'avoir le moins possible de parallélisme en mémoire distribuée.


Il existe aussi de nombreuses tentatives d'ajout du support de la localité des données à OpenMP\cite{openmp_numa}.
%
Parmi celles-ci, il y a ForestGOMP\cite{Bro10Thesis} qui propose de répartir la mémoire dès l'allocation et de placer les threads au plus près de la mémoire.
