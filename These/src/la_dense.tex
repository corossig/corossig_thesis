\subsection{Dense linear algebra}
Solving a system of linear problem is often solving a problem of type $Ax=b$, where $A$ is a matrix and $b$ and $x$ are vectors.
%
In the previous example, we already have a triangular matrix, so the solution can be found directly by solving each equation one-by-one starting with $P(X_0) = 1000$.
%
But when it's not as easy as above, other methods exist for solving these problems, like Gaussian elimination, elimination of variables or LU decomposition.


In computer science, there is many libraries for linear algebra operations.
%
The most common is BLAS\footnote{Basic Linear Algebra Subprograms} which is a set of linear algebra operations on vectors and matrices.
%
These operations are classify into 3 levels :
\begin{itemize}
  \item Level 1 : it's vectors operations (dot products, addition of two vectors, ...)
  \item Level 2 : it's matrix-vector operations (multiply a matrix by a vector, solve a system of linear equations whose coefficients are in a triangular matrix, ...)
  \item Level 3 : it's matrix-matrix operations (multiply a matrix by a matrix, ...)
\end{itemize}
%
The level of BLAS is linked to the complexity in number of operations.
%
BLAS of level 1 are bandwidth limited, there is no data reuse, each data is used only once.
%
BLAS of level 2 can reuse vector data, some optimization can be done here.
%
BLAS of level 3 have a higher complexity and some optimizations exist~\cite{blas3_opt}.

Another library for library algebra is LAPACK\footnote{Linear Algebra PACKage}, it's built on top of BLAS.
%
Operations done by BLAS and LAPACK are well optimized, for example some implementations use cache blocking technique which improve data locality and reduce cache misses.
%
Some implementations also use SIMD instructions(SSE, AVX, ...) on modern processor~\cite{intel_mkl}..
%
GPGPU implementations~\cite{nvidia_cublas} also exist as well as implementations for distributed memory~\cite{dplasma}.
%
Most of these optimizations can be done because the access pattern of BLAS operations are deterministic and some operations can be reorder without changing the final result.


Go back to the matrix of eq.~\ref{eq:ax_b}, we can see that this matrix contains a lot of zero values, and these values don't impact on the calculation.
%
One can differentiate matrices will a lot zero values from matrices with a majority of non-zeros values.
%
A matrix can be consider like a sparse matrix when the number of non-zero values is of the order of the dimension of the matrix.
%
Solving a sparse linear system use different methods than a dense linear system.
