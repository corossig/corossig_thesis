\subsection{Dense linear algebra}

Let's take a real life example to see what linear algebra is and how to use it.
%
We need to simulate the pressure of the underground which contains three different rocks.
%
Each rock has a density designed by $\rho$ and $rho(z)$ represents the density of the rock at the depth $z$.
%

%
We use the physical equation :
%
\begin{equation}
\frac{\mathrm d P}{\mathrm d z} = \rho(z)g
\end{equation}
%
where $P$ is the pressure, $z$ is the depth and $g$ is the gravitational constant.
%
We also have the boundary condition that at ground level 0, the pressure is 1000~hPa :
%
\begin{equation}
P_0 = 1000
\end{equation}
%
If we use a finite difference method at the first order
%
\begin{equation}
\frac{P_i - P_{i-1}}{z_i - z_{i-1}} = \rho(z_i)g
\end{equation}
%
If the distance between the center of all cells is the same, we can call it $\Delta{z}$ and we can replace $z_i - z_{i-1}$ by $\Delta{z}$ and multiply all terms by $\Delta{z}$ :
%
\begin{equation}
\label{eq:system_pressure}
P_i - P_{i-1} = \rho(z_i)g\Delta{z}
\end{equation}
%
\begin{equation}
\label{eq:ax_b}
\begin{bmatrix}
   1   &    0   &    0   & \cdots & \cdots & \cdots & \cdots &   0    \\
  -1   &    1   &    0   & \ddots &        &        &        & \vdots \\
   0   &   -1   &    1   &    0   & \ddots &        &        & \vdots \\
\vdots & \ddots & \ddots & \ddots & \ddots & \ddots &        & \vdots \\
\vdots &        & \ddots & \ddots & \ddots & \ddots & \ddots & \vdots \\
\vdots &        &        & \ddots &   -1   &    1   &    0   &   0    \\
\vdots &        &        &        & \ddots &   -1   &    1   &   0    \\
   0   & \cdots & \cdots & \cdots & \cdots &    0   &   -1   &   1    \\
\end{bmatrix}
*
\begin{pmatrix}
  P_0  \\
  P_1  \\
\vdots \\
\vdots \\
\vdots \\
\vdots \\
P_{i-1} \\
  p_i  \\
\end{pmatrix}
=
\begin{pmatrix}
 1000  \\
\rho(z_1)g\Delta{z}     \\
\vdots \\
\vdots \\
\vdots \\
\vdots \\
\rho(z_{i-1})g\Delta{z} \\
\rho(z_i)g\Delta{z}    \\
\end{pmatrix}
\end{equation}


We have a matrix $A$ multiply a vector $x$ equal a vector $b$.
%
By doing the multiplication we obtain exactly the system of equations \ref{eq:system_pressure}.
%
So solving a linear problem is often solving a problem of type $A*x=b$, many methods exist for solving this problem (Gaussian elimination, ...).
%
In this example, we already have a triangular matrix, so the solution can be found directly by solving each equation one-by-one starting with $P_0 = 1000$.


In computer science, there is a lot of library for doing linear algebra operations.
%
The most common is BLAS\footnote{Basic Linear Algebra Subprograms} which is a set of linear algebra operations.
%
These operations are classify into 3 categories :
\begin{itemize}
  \item Level 1 : it's vectors operations (dot products, addition of two vectors, ...)
  \item Level 2 : it's matrix-vector operations (multiply a matrix by a vector, solve a system of linear equations whose coefficients are in a triangular matrix, ...)
  \item Level 3 : it's matrix-matrix operations (multiply a matrix by a matrix, ...)
\end{itemize}
%
The level of BLAS is linked to the complexity in number of operations.
%
BLAS of Level 1 are bandwidth limited, there is no data reuse, each data is used only once.
%
BLAS of Level 2 can reuse vector data, some optimization can be done here.
%
BLAS of level 3 have a higher complexity and a lot of optimization exists.

Another library for library algebra is LAPACK\footnote{Linear Algebra PACKage}, it's build on top of BLAS.
%
Operations done by BLAS and LAPACK are well optimized, they have tilling optimization for cache blocking technique which improve data locality and reduce cache misses.
%
They also can use SIMD instructions(SSE, AVX, ...) in modern processor.
%
Some GPGPU versions also exist as well as versions for distributed memory.
%
Most of these optimizations can be done because the access pattern of BLAS operations are deterministic and some operations can be reorder without changing the final result.


Go back to the matrix of eq.~\cite{eq:ax_b}, we can see that this matrix contains a lot of zero values, and these values doesn't have too much impact for the calculation.
%
On can differentiate matrices will a lot zero values from matrices with a majority of non-zeros values.
%
A matrix can be consider like a sparse matrix when the number of non-zero values is of the order of the matrix dimension.
%
Solving a sparse linear system use different methods than a dense linear system.
