\subsubsection{Collection de matrices creuses de l'université de Floride}
La collection de matrices creuses de l'université de Floride est un ensemble de matrice provenant de diverses simulations.
%
Parmi toutes ces matrices, nous en avons choisi quatre ayant des motifs différents de ceux que nous pouvons retrouver en simulation de réservoir.
%
Nous ne nous intéressons pas aux propriétés physiques de ces simulations mais seulement aux connexions des noeuds du graphe.
%
C'est pourquoi nous allons choisir des paramètres arbitraires par rapport aux poids des tâches.


\begin{figure}[!h]
     \begin{center}
        \subfigure[Pajek/EPA]{
          \label{fig:florida_epa}
          \includegraphics[width=0.45\textwidth]{florida_epa}
        } %
        ~
        \subfigure[SNAP/roadNet-PA]{
          \label{fig:florida_roadNet}
          \includegraphics[width=0.45\textwidth]{florida_roadNet}
        }

        \subfigure[Gleich/wb-cs-stanford]{
          \label{fig:florida_wbcs}
          \includegraphics[width=0.45\textwidth]{florida_wbcs}
        }
        ~
        \subfigure[Williams/webbase-1M]{
          \label{fig:florida_webbase}
          \includegraphics[width=0.45\textwidth]{florida_webbase}
        }
    \end{center}
    \caption{Représentation du graphe de connexions des quatre matrices choisies.}
    \label{fig:florida}
\end{figure}

%   (-_-)   %
\begin{center}
  \begin{tabular}{|r|c|c|c|}
    \hline
    Nom de la & Nombre de       & Nombre de & Description\\
    matrice   & lignes/colonnes & non-zéros &  \\
    \hline
    Pajek/EPA             & 4~772     & 8~965     & \\
    SNAP/roadNet-PA       & 1~090~920 & 3~083~796 & \\
    Gleich/wb-cs-stanford & 9~914     & 36~854    & \\
    Williams/webbase-1M   & 1~000~005 & 3~105~536 & \\
    \hline
  \end{tabular}
  \captionof{table}{Descriptions des matrices utilisées.}
  \label{tab:florida}
\end{center}
