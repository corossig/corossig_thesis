\subsection{AutoNUMA}
\label{sec:autonuma}
Linux n'a pas adopté la politique d'allocation Next touch.
%
\`A la place, les développeurs de Linux ont choisi d'implémenter un autre mécanisme pour améliorer la localité NUMA : {\em AutoNUMA}.
%
Ce mécanisme va analyser périodiquement une portion de la mémoire d'un processus, et de la même façon que la politique next touch, le prochain accès à une page de cette portion entrainera un déplacement de la page.
%
Pour détecter l'accès à une page, le noyau utilise la MMU.
%
Il supprime la relation adresse virtuelle vers adresse physique de la MMU et peut ainsi recevoir un signal lors du prochain accès.
%
Ce mécanisme à un surcoût, c'est pourquoi il n'est pas appliqué sur toute la mémoire d'un coup.
%
Le noyau donne la possibilité de modifier plusieurs paramètres de ce mécanisme, mais ces paramètres sont globaux.
%
Parmi ces paramètres, il y a {\em scan\_delay} et {\em scan\_size}.
%
Tous les ``scan\_delay'', les ``scan\_size'' pages suivantes sont traitées.
%
Une fois arrivé au bout de l'espace d'adressage, le scanner recommence au début de l'espace d'adressage.
%
La variable scan\_delay change de valeur en fonction du nombre de pages déplacées.
%
Elle diminue quand il y a beaucoup de fautes NUMA et augmente quand les pages sont bien placées.
%
Ainsi, une application dont les threads accèdent toujours à la même mémoire aura automatiquement un bon placement mémoire.

Ce mécanisme comporte plusieurs défauts.
%
Sa configuration est globale au système, on ne peut pas l'activer que pour un processus particulier.
%
Les paramètres ne peuvent être définis que par l'administrateur de la machine.
%
Le mécanisme s'applique sur toute la mémoire, même les zones mémoire peu utilisées.
%
Dans le cas de notre application, cette politique permet d'améliorer la localité mémoire sans changer une ligne de code.
%
Mais le surcoût lié à l'analyse du placement des pages mémoire peut devenir problématique.
