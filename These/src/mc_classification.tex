\subsection{Classification of runtime}
Not all runtimes are equal, we can try to classify all these runtime by functionality.
%
We can consider three form of parallelism expression : loop parallelism, PTG and insert task paradigm.
%



%% The first table~\ref{tab:runtime_family} summarizes capabilities of a set of runtime, a {\it ++} entry means that the capacity is often put forward in publication, a single {\it +} means that the runtime has the functionality but it is not a major advantage of the runtime.
%% %
%% The second table~\ref{tab:runtime_archi} summarizes target architecture of the same set of runtime.
%% %
%% In the case of distributed memory, we can see two method to address this problem.
%% An implicit method when the runtime has a memory manager and can do automatic transfer.
%% Or an explicit method often describe as user task, some runtime, like StarPU, support asynchronous transfer, without this support the explicit method is less efficient.


%% We can also try to differentiate other differences, like the use of an API ({\it Application Programming Interface}) or the use of a source-to-source compiler.
%% %
%% Data management is also important in a runtime to optimize data transfer between CPU and GPU, or between two CPUs.

%% \Beginommen{table}[h!]
%% \centering
%% \begin{tabular}{c|ccc}
%%   \textit{runtime}& loop parallelism & PTG & Insert task\\
%%   \hline
%%         Cilk           &    &    & ++ \\
%%         Cilk++         & ++ &    & ++ \\
%%         OpenMP $<$ 3.0 & ++ &    &    \\
%%         OpenMP 3.x     & ++ &    & +  \\
%%         OpenMP 4.0     & ++ &    & +  \\
%%         OpenACC        & ++ &    & +  \\
%%         Intel TBB      & +  &    & ++ \\
%%         OMPSs          & +  &    & ++ \\
%%         Intel CnC      & +  & ++ &    \\
%%         PaRSEC         &    & ++ &    \\
%%  PGAS(coarray fortran) & ++ &    &    \\
%%         StarPU         &    &    & ++ \\
%%         KAAPI          & ++ &    &    \\
%%         X-KAAPI        & +  &    & ++
%% \end{tabular}
%% \caption{Classification of runtime by capabilities}
%% \label{tab:runtime_family}
%% \end{table}

%% \begin{table}[h!]
%% \centering
%% \begin{tabular}{c|ccc}
%%   \textit{runtime} & Shared memory & Distributed memory & GPU accelerator \\
%% \hline
%%         Cilk           & X &           &   \\
%%         Cilk++         & X &           &   \\
%%         OpenMP $<$ 3.0 & X &           &   \\
%%         OpenMP 3.x     & X &           &   \\
%%         OpenMP 4.0     & X &           & X \\
%%         OpenACC        & X &           & X \\
%%         Intel TBB      & X &           &   \\
%%         OMPSs          & X & explicit  & X \\
%%         Intel CnC      & X & implicit  &   \\
%%         PaRSEC         & X & implicit  &   \\
%%  PGAS(coarray fortran) & X & implicit  &   \\
%%         StarPU         & X & implicit/explicit  & X \\
%%         KAAPI          & X &           & X \\
%%         X-KAAPI        & X & explicit  & X
%% \end{tabular}
%% \caption{Classification of runtime by supported architecture}
%% \label{tab:runtime_archi}
%% \end{table}
