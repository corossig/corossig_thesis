\subsection{Solutions actuelles}
Le problème de granularité de la programmation à base de tâche est un problème bien connu et est étudié depuis un long moment.
%
Certains ordonnanceurs à base de tâche essaient de résoudre ce problème en utilisant différentes approches.
%
Par exemple, X-Kaapi~\cite{xkaapi} a introduit le concept de tâches divisibles aussi appelé {\em adaptive task model} et fonctionne de la manière suivante :
%
quand un travailleur passe dans l'état d'attente, il émet une requête de travail à un autre travailleur.
%
Les autres travailleurs, qui sont dans l'état travail, doivent vérifier régulièrement s'ils ont reçu une requête de vol.
%
Puis pour traiter cette requête, le travail restant est divisé en deux, la fonction divisant le travail en deux doit être écrite par le programmeur, ce n'est pas automatique.
%
Cette fonction peut être triviale dans le cas d'une boucle for parallèle ou dans le cas de parallélisme sous la forme d'arbre.
%
Mais dans le cas général, il n'est pas toujours possible de diviser un DAG en deux DAG totalement indépendants.


Une autre approche possible, comme donnée par Capsules\cite{capsules}, requiert que le programmeur définisse plusieurs granularités.
%
L'ordonnanceur pourra ensuite choisir la granularité qui s'adapte le mieux à la situation.
%
Le programmeur doit donc architecturer son application de façon à pouvoir avoir plusieurs granularités, ce qui peut dans certains cas être difficile à exprimer de manière abstraite.
