\subsection{Factorisation ILU(k) avec renumérotation des cellules}
Lorsque nous renumérotons les cellules, nous obtenons un graphe de tâches différents.
%
Une numérotation rouge-noir donne un graphe très large avec seulement 2 niveaux de hauteur.
%
Nous pouvons aussi utiliser une numérotation de type {\em nested dissection} pour obtenir un autre type de graphe.
%
Ce graphe aura la forme d'un arbre.
%
De plus, augmenter le niveau de remplissage de l'algorithme ILU(k) créera de nouvelles connexions dans le graphe.
%
Nous allons donc utiliser notre simulateur avec un numérotation nested dissection des matrices.
%
Nous définissons les paramètres 0.6 pour les effets caches et 2 pour le coût d'ordonnancement.

