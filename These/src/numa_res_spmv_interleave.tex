\subsubsection{Interleave}
Pour diminuer les effets NUMA, nous pouvons utiliser la politique d'allocation interleave.
%
Cette politique va distribuer uniformément les pages mémoires sur les différents bancs NUMA.
%
Nous allons donc augmenter la bande passante mémoire en ne modifiant que la latence mémoire moyenne.
%
Sur Rostand, nous obtenons un gain de performance d'environ 20~\% mais les performances sont toujours en dessous des performances obtenues en mémoire distribuée (Fig.~\ref{fig:res_spmv_inter_rostand}).



Sur Manumanu, on obtient de bons résultats jusqu'à 16 coeurs (Fig.~\ref{fig:res_spmv_inter_manu}).
%
Au delà, nous commençons à utiliser le SGI$^\registered$ NUMAlink$^{\rm TM}$\cite{numalink} et les temps de latence des accès mémoire augmentent.
%
En effet, la majorité des accès mémoire se font sur des bancs NUMA distants.
%
Au final, les résultats de l'allocation interleave sur Manumanu sont proches des résultats de l'allocation first touch.
%
Ce n'est donc pas la bonne solution pour exploiter les performances de cette machine.
