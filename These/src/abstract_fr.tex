La résolution de grands systèmes linéaire creux est un élément essentiel des simulations numériques. Ces résolutions peuvent représenter jusqu'à 80\% du temps de calcul des simulations.
Une parallélisation efficace des noyaux d'algèbre linéaire creuse conduira donc à obtenir de meilleurs performances. En mémoire distribuée, la parallélisation de ces noyaux se fait le plus souvent en modifiant le schéma numérique. Par contre, en mémoire partagée, un parallélisme plus efficace peut être utilisé. Il est donc important d'utiliser deux niveaux de parallélisme, un premier niveau entre les noeuds d'une grappe de serveur et une deuxième niveau à l'intérieur du noeud. Lors de l'utilisation de méthodes itératives en mémoire partagée, les graphes de tâches permettent de décrire naturellement le parallélisme en prenant comme granularité le travail sur une ligne de la matrice. Malheureusement, cette granularité est trop fine et ne permet pas d'obtenir de bonnes performances.
Dans cette thèse, nous allons étudier le problème de la granularité pour la parallélisation par graphe de tâches. Nous proposerons d'augmenter la granularité des tâches de calcul en créant des agrégats de tâches qui deviendront eux-mêmes des tâches. L'ensemble de ces agrégats et des nouvelles dépendances entre les agrégats formera un graphe de granularité plus grossière. Ce graphe sera ensuite utilisé par un ordonnanceur de tâches pour obtenir de meilleurs résultats. Nous utiliserons comme exemple la factorisation ILU d'une matrice et nous montrerons les améliorations apportées par cette méthode. Dans un second temps, nous nous concentrerons sur les machines à architecture NUMA. Dans le cas de l'utilisation d'algorithmes limités par la bande passante mémoire, il est intéressant de réduire les effets NUMA liés à cette architecture. Nous montrerons comment prendre en compte ces effets dans un intergiciel à base de tâches pour améliorer les performances d'un programme.

Mots-clés : parallélisme, graphe de tâches, supports d’exécution, NUMA, multi-coeurs, algèbre linéaire creuse
