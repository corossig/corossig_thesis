La résolution de grands systèmes linéaires creux est un élément essentiel des simulations numériques. Ces résolutions peuvent représenter jusqu'à 80\% du temps de calcul des simulations.
Une parallélisation efficace des noyaux d'algèbre linéaire creuse conduira donc à obtenir de meilleures performances. En mémoire distribuée, la parallélisation de ces noyaux se fait le plus souvent en modifiant le schéma numérique. Par contre, en mémoire partagée, un parallélisme plus efficace peut être utilisé. Il est donc important d'utiliser deux niveaux de parallélisme, un premier niveau entre les noeuds d'une grappe de serveur et une deuxième niveau à l'intérieur du noeud. Lors de l'utilisation de méthodes itératives en mémoire partagée, les graphes de tâches permettent de décrire naturellement le parallélisme en prenant comme granularité le travail sur une ligne de la matrice. Malheureusement, cette granularité est trop fine et ne permet pas d'obtenir de bonnes performances à cause du surcoût de l'ordonnanceur de tâches.
Dans cette thèse, nous étudions le problème de la granularité pour la parallélisation par graphe de tâches. Nous proposons d'augmenter la granularité des tâches de calcul en créant des agrégats de tâches qui deviendront eux-mêmes des tâches. L'ensemble de ces agrégats et des nouvelles dépendances entre les agrégats forme un graphe de granularité plus grossière. Ce graphe est ensuite utilisé par un ordonnanceur de tâches pour obtenir de meilleurs résultats. Nous utilisons comme exemple la factorisation LU incomplète d'une matrice creuse et nous montrons les améliorations apportées par cette méthode. Puis, dans un second temps, nous nous concentrons sur les machines à architecture NUMA. Dans le cas de l'utilisation d'algorithmes limités par la bande passante mémoire, il est intéressant de réduire les effets NUMA liés à cette architecture en plaçant soi-même les données. Nous montrons comment prendre en compte ces effets dans un intergiciel à base de tâches pour ainsi améliorer les performances d'un programme parallèle.

Mots-clés : parallélisme, graphe de tâches, supports d'exécution, NUMA, multi-coeurs, algèbre linéaire creuse
