\subsection{Exemples de stratégies}
\url{http://view.eecs.berkeley.edu/wiki/Dwarf_Mine}

Le choix des opérateurs dans Taggre se fait à l'appréciation du programmeur.
%
Les {\em 13 nains de Berkeley} est une méthode de classification des problèmes suivant leurs motifs de calculs et de communication.
%
Il est intéressant de voir si Taggre peut répondre aux 13 problèmes et si c'est le cas quelle serai la stratégie d'agrégation à utiliser.

\subsubsection{Algèbre linéaire dense}
Les graphes utilisés en algèbre linéaire dense sont très régulier.
%
Une grande partie des optimisations de graphe peut être faite statiquement dans le code.
%
L'utilisation de Taggre reste toutefois possible, mais donnera des résultats moins bon.
%
L'opérateur D fonctionne très bien avec les problèmes réguliers.
%
Mais le plus efficace serai d'écrire un opérateur qui cherche des motifs pour les agréger en une seule tâche.


\subsubsection{Algèbre linéaire creuse}
Taggre a été conçu pour répondre aux problèmes de l'algèbre linéaire creuse.
%
Les techniques d'agrégations dépendront du problème à résoudre.
%
Dans notre cas, nous avons des modèles 3D, nous allons donc utiliser l'opérateur C.
%
Si le graphe n'est toujours pas assez grossier, nous pouvons ajouter l'opérateur D avec un paramètre assez petit.
%
Dans d'autre cas, il est nécessaire d'analyser le graphe mais généralement l'opérateur D avec un paramètre assez gros fonctionne.



\subsubsection{Méthodes spectrales}
Les méthodes spectrales sont composées de plusieurs paliers, les tâches d'un même palier sont indépendantes.
%
Les dépendances entre les paliers sont croisées.
%
L'opérateur F avec comme paramètre 2 à 3 fois le nombre de threads devrait limiter le parallélisme sans toutefois trop impacter l'équilibrage de charge.



\subsubsection{Méthodes N-Body}
Ici, il s'agit de simuler un ensemble de particules qui se déplacent.
%
Il n'y a pas de graphe statique de tâches, Taggre n'est pas efficace pour ce genre de méthodes.



\subsubsection{Grilles structurées}
Le problème de parallélisation des problèmes avec des grilles structurées peut se résoudre avec du parallélisme de boucle.
%
\`A chaque itération, chaque cellule est mise à jour en fonction des cellules adjacentes.
%
Si nous souhaitons utiliser Taggre, nous pouvons utiliser l'opérateur F pour grossir le grain de calcul.


\subsubsection{Grilles non-structurées}
Les solutions à ce problème sont les mêmes que pour les grilles structurées.
%
Il vaut mieux utiliser du parallélisme de boucle avec un ordonnanceur dynamique pour avec un bon équilibrage de charge.
%
Avec Taggre, il faut définir le poids de chaque tâche et utiliser l'opérateur F qui formera des tâches plus grossières de même poids.


\subsubsection{MapReduce}
Les problèmes de type MapReduce se parallélisent très bien à condition que la fonction de distribution soit correcte.
%
Il n'y a donc aucun intérêt à utiliser Taggre.



\subsubsection{Logique combinatoire}
Il s'agit de parallélisme pouvant seulement être résolu au moment de la conception des processeurs.
%
Le parallélisme est vraiment trop fin, Taggre ne peut rien faire.


\subsubsection{Graph Traversal}
Il n'a pas de graphe de tâches, les étapes sont totalement séquentielles.


\subsubsection{Dynamic Programming}
Le problème se présente sous la forme d'un arbre.
%
Il faut donc utiliser l'opérateur F suivi de l'opérateur S.


\subsubsection{Backtrack and Branch-and-Bound}
Le graphe n'est pas connu à l'avance, il est impossible d'utiliser Taggre.


\subsubsection{Graphical Models}
Il n'a pas de graphe de tâches, les étapes sont totalement séquentielles.


\subsubsection{Finite State Machines}
Les machines à états sont composées de graphes directs, mais ces graphes peuvent contenir des cycles.
