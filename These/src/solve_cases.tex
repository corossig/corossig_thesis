\subsection{Cas d'étude}
Pour être en mesure de tester notre méthode de parallélisation en mémoire partagée, nous utilisons un code de solveur linéaire développé à Total SA.
%
Nous allons essayer de paralléliser la partie GMRES préconditionné du code.
%
Dans le but d'évaluer le gain de performance, nous avons choisi des systèmes linéaires à résoudre avec le solveur linéaire.


Ces systèmes linéaires sont représentés sous la forme d'une matrice et d'un vecteur second membre.
%
La structure des matrices est dépendante du problème simulé.
%
Nous utilisons un maillage structuré avec un schéma de discrétisation en 7 points (e.g., volume fini).
%
En gardant une numérotation naturelle, nos matrices aurons donc une structure composée de sept diagonales.
%
En fonction de ce que nous voulons simuler, les entrées de la matrice pourront être scalaires ou composé de petits blocs.
%
Si nous ne simulons que la pression, nous aurons des entrées scalaires.
%
Les simulations de type {\em black-oil} sont les plus utilisées en simulation de réservoir.
%
Il s'agit de simuler 3 variables primaires, la concentration en huile, en gaz et en eau de chaque cellule.
%
Il arrive aussi que l'on souhaite simuler plus précisément les différents types d'huiles contenues dans les réservoirs.
%
Dans ce cas, nous utiliserons un modèle compositionnel dans lequel chaque variable primaire correspondra à la saturation d'un type d'hydrocarbure.
%
Pour les cas black-oil et les cas compositionnels, les entrées de la matrice seront de petits blocs denses de taille $npri*npri$ où $npri$ est les nombres de variables primaires.

Pour évaluer notre code, nous allons utiliser le cas test SPE10 qui est basé sur les données prises du second modèle du 10ème cas test SPE\cite{SPE10}.
%
C'est un réservoir de 1~122~000 de cellules, organisées dans une grille 3D cartésienne de taille 60~x~220~x~85.
%
Il s'agit d'un modèle black-oil, donc à 3 variables primaires, et c'est un problème de référence dans l'industrie pétrolière.
%
Les autres cas tests seront générés par un programme développé en interne, nous utiliserons un cas pression, un cas black-oil et un cas compositionnel à 8 composants.
%
Ce programme génère des cubes 3D cartésiens de taille arbitraire.
%
Ces cas générés nous permettent de tester différentes combinaisons de tailles dans le but d'évaluer le passage à l'échelle de nos algorithmes.

La partie GMRES du code que nous souhaitons paralléliser est composée de plusieurs noyaux d'algèbre linéaire creux.
%
Il y a la factorisation ILU et les résolutions triangulaires, le produit matrice vecteur creux et le produit scalaire.
%
Le parallélisme exploitable dans ces noyaux est différent, il peut être plus ou moins difficile à exploiter.
%
Dans le cas du produit matrice vecteur creux, la multiplication de chaque ligne de la matrice est indépendante, le parallélisme s'exploite facilement.
%
De même pour le produit scalaire, chaque élément du vecteur peut être traité indépendamment.
%
Pour la factorisation ILU c'est différent, certaines lignes doivent être factorisées avant d'autres, le parallélisme est donc plus dur à exploiter.
%
Les résolutions triangulaires se parallélisent de la même façon que la factorisation ILU.
%
Dans la suite de la thèse, nous expliquerons comment exploiter efficacement le parallélisme dans ces quatre cas.
