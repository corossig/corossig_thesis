\section{Conclusion}
La simulation de réservoir permet aux compagnies pétrolières d'optimiser le placement des puits lors de l'exploitation d'un champ pétrolier.
%
La résolution des équations physiques peut prendre jusqu'à 80\% du temps de simulation.
%
Nous nous sommes donc concentrés à améliorer cette partie en commençant par la factorisation ILU, un des préconditionneurs du GMRES.
%
La parallélisation de la factorisation ILU(k) s'exprime facilement sous la forme d'un graphe de tâches, mais la granularité des tâches ne permet pas d'obtenir de bonnes performances.
%
La plupart des ordonnanceurs actuels mettent plus de temps à ordonnancer une tâche que la tâche met à exécuter son code.
%
Pour pouvoir modifier facilement cette granularité, nous avons élaboré un nouveau cadriciel que nous avons appelé Taggre.
%
Ce cadriciel prend en entrée un graphe de tâches à grain très fin et utilise des algorithmes pour créer de nouveaux groupes de tâches.
%
Pour cela, il utilise des opérateurs d'agrégations que nous avons définies dans le chapitre 2.
%
Chaque opérateur aura un objectif particulier d'optimisation du graphe.
%
En combinant ces opérateurs, nous pouvons obtenir un nouveau graphe à grain grossier.
%
Ce graphe pourra être ensuite utilisé par un ordonnanceur de tâches.

Cette méthode d'agrégation de tâches est générique.
%
Dans le cadre de cette thèse, nous l'avons appliqué à la factorisation ILU(k) ainsi qu'aux résolutions triangulaires associées.
%
Mais cette méthode pourrait aussi s'appliquer à d'autres noyaux d'algèbre linéaire creuse ainsi qu'à d'autres genres de problèmes représentant le parallélisme sous la forme d'un graphe de tâche.

Nous nous sommes ensuite concentrés sur l'optimisation des accès mémoires.
%
L'architecture de type NUMA des machines que nous utilisons nous a conduits à créer un nouvel ordonnanceur prenant en compte la localité mémoire des tâches de calcul.
%
Nos algorithmes étant limités par la bande passante mémoire, l'amélioration de la localité mémoire a conduit à une amélioration directe des performances.
%
Malheureusement, cet ordonnanceur a été développé avec pour objectif de fonctionner sur des machines avec 2 bancs NUMA.
%
L'utilisation d'une machine ayant plus de bancs NUMA a montré les limites de cet ordonnanceur.
%
Malgré ces limites, les résultats obtenus restent meilleurs que ceux obtenus avec un ordonnanceur classique.
%
Les résultats de ces travaux ont été présentés à dans le workshop PDSEC de la conférence IPDPS2013.

Durant cette thèse, nous avons aussi eu l'occasion d'essayer notre code de calcul sur un coprocesseur Intel Xeon Phi.
%
La bande passante mémoire de ces coprocesseurs étant plus élevée que celle des processeurs que nous utilisons, nous avons obtenu de très bonnes performances sur les principaux noyaux de calculs du code.
%
Par contre, l'utilisation du mode natif pour la partie séquentielle du code contrebalance les gains obtenus.
