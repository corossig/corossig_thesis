\section{Conclusion}
La parallélisation de la factorisation ILU(k) s'exprime facilement sous la forme d'un graphe de tâches mais la granularité des tâches ne permet pas d'obtenir de bonnes performances.
%
Pour pouvoir modifier facilement cette granularité, nous avons élaboré un nouveau cadriciel.
%
Ce cadriciel prend en entrée un graphe de tâches à grain très fin et utilise des algorithmes pour créer des nouveaux groupes de tâches.
%
Pour cela, il utilise des opérateurs d'agrégations que nous avons défini dans le chapitre 2.
%
Chaque opérateur aura un objectif d'optimisation du graphe particulier.
%
En combinant ces opérateurs, nous pouvons obtenir un graphe à grain grossier.
%
Ce graphe pourra être ensuite utilisé par un ordonnanceur de tâches.


Cette méthode d'agrégation de tâches est générique.
%
Dans le cadre de cette thèse, nous l'avons appliqué à la factorisation ILU(k) ainsi qu'aux résolutions triangulaires associées.
%
Mais cette méthode pourrait aussi s'appliquer à d'autres noyaux d'algèbre linéaire creuse ainsi qu'à d'autre genre de problème représentant le parallélisme sous la forme d'un graphe de tâche.


L'architecture de type NUMA des machines que nous utilisons, nous a conduit à créer un nouvel ordonnanceur prenant en compte la localité mémoire des tâches de calcul.
%
Nos algorithmes étant limités par la bande passante mémoire, l'amélioration de la localité mémoire a conduit à une amélioration directe des performances.
%
Malheureusement, cet ordonnanceur a été développé avec pour objectif de fonctionner sur des machines avec 2 bancs NUMA.
%
L'utilisation d'une machine ayant plus de bancs NUMA a montré les limites de cet ordonnanceur.
%
Malgré ces limites, les résultats obtenues reste meilleurs que ceux obtenus avec un ordonnanceur classique.
