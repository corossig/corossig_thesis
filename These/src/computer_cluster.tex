\subsection{Grappe de calcul}
Pour être capable de simuler de grands problèmes physiques, nous avons besoin de beaucoup de puissance de calcul.
%
Cette puissance se mesure en FLOPS\footnote{FLoating-point Operations Per Second}, il s'agit du nombre d'opérations par seconde qu'un ordinateur peut effectuer sur des nombres à virgules flottantes.
%
Depuis déjà quelque temps ces simulations sont faites sur des grappes de calcul.
%
Les grappes de calcul sont composées de noeuds, similaires aux ordinateurs de bureau, connectés par un réseau à faible latence/haut débit.
%
Chaque noeud a sa propre mémoire, fait tourner son propre système d'exploitation et peut être considéré comme une machine isolée.
%
La principale difficulté pour écrire des programmes pour les grappes de calcul vient de la mémoire distribuée. TODO continue
