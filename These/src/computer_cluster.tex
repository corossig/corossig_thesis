\subsection{Grappe de serveurs}
Au final, il est possible de connecter plusieurs ordinateurs entre eux pour obtenir une machine encore plus puissante.
%
Chaque ordinateur est appelé noeud de calcul, il a sa propre mémoire, fait tourner son propre système d'exploitation et peut être considéré comme une machine isolée.
%
Les noeuds sont reliés entre eux par un réseau à faible latence/haut débit, tel que Infiniband ou Myrinet.
%
Le principal avantage de cette solution est le passage à l'échelle.
%
Il est possible de construire des machines de très grandes tailles et très puissantes.



Parmi les 500 machines les plus puissantes au monde au moment de l'écriture de cette thèse, 429 sont des grappes de serveurs.
%
La machine la plus puissante est la {\em TIANHE-2} avec une puissance crête d'environ 55~PFLOPS.
%
Mais l'utilisation de ces machines pose un sérieux problème, elles ne sont pas vraiment faciles à programmer.
%
Il faut prendre en compte que la mémoire n'est pas globale, chaque noeud ne voit que sa mémoire locale.
