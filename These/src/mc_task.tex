\subsection{Parallélisme à base de tâches}
La théorie des graphes est une théorie utilisée en informatique pour résoudre de nombreux problèmes.
%
% Un graphe est un ensemble de n{\oe}uds et certains de ces n{\oe}uds sont reliés par des arêtes pouvant être orientées.
%
% Un chemin est un ensemble d'arêtes qui relie deux n{\oe}uds ensemble, un cycle est un chemin qui permet de relier un n{\oe}ud à lui-même.
% % ST: Est ce vraiment utile de définir les graphes ?
% Si un graphe orienté ne contient pas de cycle, on dit qu'il est acyclique.
%
Pour représenter le parallélisme à base de tâches, nous utilisons des graphes orientés acycliques, ou DAG\footnote{Directed Acyclic Graph}.
%
Dans ce modèle, les n{\oe}uds du graphe représentent une action à effectuer et les arêtes représentent l'ordre séquentiel entre deux actions.
%
Ainsi, nous pouvons représenter le parallélisme sous une forme abstraite indépendamment des ressources matérielles disponibles.
%
Il existe de nombreux travaux appliquant ce principe et permettant ainsi de paralléliser efficacement des codes de calcul~\cite{BBAC2014,LSAT2013,LY2012,ABGL2013}.
%
La démocratisation des processeurs multi-coeurs a engendrée l'apparition de cadriciels à base de tâches~\cite{taskscomparison} comme illustré par des outils tel que Intel TBB~\cite{Intel_TBB} et que le support des tâches dans OpenMP~3.0~\cite{openmptasks} pour du parallélisme multi-coeur, ou StarSs/OmpSs~\cite{OMPSs} ainsi que StarPU~\cite{starpu} et X-Kaapi~\cite{xkaapi} pour le support de plateformes hétérogènes.
%
Dans ces modèles, la granularité correspond au nombre d'instructions processeur contenues dans la tâche.
%
Cette granularité dépend de l'algorithme à paralléliser ainsi que de l'ordonnanceur de tâches.
