\subsection{Task paradigm}
In computer programming, we can use tasks model to express the potential application parallelism in an abstract way.
%
It is usually represented under a DAG\footnote{Direct Acyclic Graph} form, nodes are computational part and edges are dependencies between nodes.
%
This model is independent from the available hardware resources and was once popularized by tools such as Cilk~\cite{Cilk} appeared in 1994.
%
The widespread availability of multi-core processors recently revived the popularity of task scheduling frameworks~\cite{taskscomparison} as exemplified by tools such as Intel TBB~\cite{Intel::TBB} and OpenMP~3.0's task support~\cite{openmptasks} for regular multi-core platforms, or StarSs/OmpSs derivatives~\cite{OMPSs} as well as StarPU~\cite{starpu} and X-Kaapi~\cite{xkaapi} for heterogeneous platforms.
%
In these model, granularity refers to the number of CPU instructions inside a task.
%
The tasks grain size depends on the algorithm to parallelize and also of the runtime task scheduler.
