\subsection{Multiplication matrice vecteur creuse}
La multiplication du matrice creuse par un vecteur est une opération dont le ratio nombre d'opérations par le nombre d'octet lus est petit.
%
Dans le cas d'une matrice scalaire, ce ratio vaut environ $1/10$ en double précision.
%
Pour chaque valeur non-nulles de la matrice, il faut lire cette valeur, l'indice de la colonne et la valeur contenue dans le vecteur à l'indice de la colonne.
%
Il faut ensuite multiplier les deux valeurs ensemble et l'ajouter à un accumulateur, ce qui fait 2 opérations pour 20 octets lus.
%
Si nous utilisons trois variables primaires, chaque entrée de la matrice est un bloc 3 par 3.
%
Nous devons donc lire ce bloc (9*8 Octets), lire l'indice de colonne (4 Octets) et finalement lire 3 valeurs dans le vecteur (3*8 Octets).
%
Pour chaque valeurs du bloc nous avons 2 opérations à faire (2*9), nous avons donc un ratio de $18/100$ soit environ $1/5,5$.
%
Avec huit variables primaires, le ratio est environ de $1/4,5$.



Le roofline model nous montre que nous sommes vraiment limité par la bande passante.
%
On peux aussi voir
