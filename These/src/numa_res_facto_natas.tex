\subsubsection{NATaS}
\`A la différence des autres ordonnanceurs, NATaS va tenir compte de l'affinité NUMA des tâches.
%
Cette affinité a été définit par Taggre de tel sorte à équilibrer la charge sur les différents bancs NUMA.
%
NATaS offre de meilleurs performances sur Rostand par rapport à la politique d'allocation interleave.
%
La factorisation est 40\% plus rapide avec 8 variables primaires et la résolution triangulaire est 23\% plus rapide.
%
Avec 1 variable primaire nous n'obtenons pas de gain sur la résolution triangulaire.


%   (-_-)   %
\begin{figure}[!ht]
     \begin{center}
        \subfigure[Factorisation ILU(0).]{%
          \label{fig:res_facto_nas_rostand}
          \includegraphics[width=0.49\textwidth]{res_facto_nas}
        }%
        \subfigure[Résolution triangulaire]{%
          \label{fig:res_trsv_nas_rostand}
          \includegraphics[width=0.49\textwidth]{res_trsv_nas}
        }%
    \end{center}
    \caption{Performances sur Rostand avec NATaS.}
\end{figure}
