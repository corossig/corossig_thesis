Taggre a été conçu pour répondre aux problèmes rencontrés avec les graphes de tâches issues de la simulation de réservoir.
%
Mais il peut aussi être utilisé sur des graphes de tâches provenant d'autres types de problèmes.
%
Nous allons dans un premier temps évaluer les heuristiques utilisées par Taggre sur des problèmes utilisant des graphes de tâches.
%
Puis nous étudierons les résultats que nous obtenons lors de la simulation de réservoir.


Pour évaluer l'amélioration apportée par chaque heuristique, nous avons intégré dans Taggre un simulateur minimal qui estimera le temps d'ordonnancement du graphe.
%
Cette estimation est essentielle dans la mesure où elle permet de mesurer le parallélisme restant pouvant être extrait du graphe grossier.
%
Pour modéliser le plus fidèlement possible les améliorations de performances liées à l'agrégation, nous prenons en compte deux paramètres.
%
Le premier paramètre est le surcoût lié à l'ordonnancement de la tâche, il s'agit du temps à ajouter au début de chaque exécution d'une tâche.
%
Le deuxième paramètre correspond à l'amélioration des effets de cache, dans notre cas il s'agit du temps gagné lorsque deux lignes consécutives de la matrice sont factorisées dans la même tâche.
%
Ce paramètre prend la forme d'un nombre réel entre 0 et 1 qui correspond au pourcentage de temps d'exécution d'une tâche quand celle-ci est exécutée juste après une tâche dont elle peut bénéficier des effets de cache.
%
Ces deux paramètres sont dépendants du problème traité.
%
Dans les différents cas tests que nous essayerons, nous choisirons arbitrairement ces valeurs pour essayer différentes agrégations.
