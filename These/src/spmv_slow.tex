\subsection{Deux fois plus de coeurs mais pas deux fois plus rapide}
Comme dit dans le chapitre précédent, l'algorithme du GMRES se parallélise bien parce qu'il est essentiellement composé d'opérations sur les vecteurs.
%
L'opération la plus coûteuse est le produit d'une matrice par un vecteur (SpMV).
%
Notre implémenetation du SpMV est optimisé pour prendre en compte la structure bloc des entrées de la matrice quand le nombre de variables primaires est supérieure à 1.
%
Dans ce cas, nous pouvons réutiliser des données en cache.
%
Malgré ces optimisations, le SpMV est toujours limité par la bande passante mémoire.
%
Les courbes de scalabilité du SpMV nous montre que le gain de performance n'est linéaire avec le nombre de coeurs.
%
Le constat est même pire que ça, on arrive difficilement à un speedup de 2 sur 12 coeurs lorsqu'on utilise une seule variable primaire.
%
Si on regarde les compteurs matériels, on s'aperçoit que la moitié de la bande passante de chaque banc NUMA est utilisée par des accès distant.
