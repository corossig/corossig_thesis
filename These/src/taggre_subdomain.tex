\subsection{Décomposition en sous-domaines}
Le parallélisme dans le partie solveur linéaire du code peut être obtenu de deux façons différentes.
%
Jusqu'à maintenant, nous avons étudier le parallélisme avec des paradigmes de parallélisation en mémoire partagée, mais on peut aussi utiliser une parallélisation par passage de messages.
%
Dans ce cas, on change le préconditionneur et donc le résultat.
%
On va utiliser la décomposition de domaine pour distribuer le travail entre les différents processus.
%
Chaque processus traitera un domaine qui correspond un sous-ensemble des cellules du réservoir.
%
Le nouveau préconditionneur sera toujours une factorisation ILU mais seulement sur les sous ensemble de cellules.
%
Ici le parallélisme est idéal, chaque processus pourra exécuter une factorisation ILU parallèlement aux autres processus.
%
Malgré ce parallélisme, nous n'obtenons pas un speed parfait.
%
L'explication se situe au niveau de la mémoire, on est limité par la bande passante.
%
C'est un problème récurrent quand on fait de l'algèbre linéaire.
%
Si on compare les speedups obtenus avec ceux obtenues avec l'agrégation de tâche on peux voir des résultats légèrement meilleurs mais il ne faut pas oublier que le résultat numérique est moins bon et ce dernier critère est très dur à évaluer.
