\subsection{Surcoût d'agrégation}
L'application des opérateurs d'agrégation a un surcoût.
%
Il est nécessaire de savoir à partir de quel moment le surcoût d'agrégation devient plus petit que le temps gagné par l'agrégation.
%
Dans notre cas l'agrégation est faite au début du programme et reste valide tant que la structure du réservoir ne change pas.
%
Le temps passé à appliquer les opérateurs d'agrégation va dépendre du nombre de tâches, des opérateurs et de l'ordre des opérateurs.
%
L'opérateur C mettra 1,5~$\mu{s}$ à traiter une tâche, l'opérateur F mettra 1,9~$\mu{s}$ et l'opérateur D mettra 2,5~$\mu{s}$.
%
Par exemple pour la matrice SPE10 il faut 1,67~s pour appliquer l'opérateur C, 2,94~s pour appliquer l'opérateur D(8) et 2.19s pour appliquer CD(2).
%
Toujours pour le cas SPE10, il devient rentable d'utiliser Taggre à partir d'environ 8 factorisations ou 10 résolutions triangulaires (Tab.~\ref{tab:gain_agg}).
%
Les 10 résolutions triangulaires peuvent être faites dans une résolution du GMRES.



%   (-_-)   %
\begin{center}
  \begin{tabular}{ | r | c || c | c | c | }
    \hline
    Matrice & Agrégation & Temps agrégation (s) & Gain factorisation (s) & Gain résolution (s)\\
    \hline
    \hline
    SPE10   &      C     & 1,67          & 0,188          & 0,171 \\
    \hline
    SPE10   &    D(8)    & 2,94          & 0,087          & 0,124 \\
    \hline
    SPE10   &    CD(2)   & 2,19          & 0,189          & 0.173 \\
    \hline
    \hline
    Cube 100&      C     & 1,49          & 0,167          & 0,163 \\
    \hline
    Cube 100&    D(8)    & 2,52          & 0,095          & 0,132 \\
    \hline
    Cube 100&    CD(2)   & 2,52          & 0,167          & 0,165 \\
    \hline
  \end{tabular}
  \captionof{table}{Temps d'application des opérateurs d'agrégation et les gains de temps obtenus sur les opérations d'algèbre linéaire.}
  \label{tab:gain_agg}
\end{center}
