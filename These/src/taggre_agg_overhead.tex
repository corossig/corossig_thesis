\subsection{Surcoût d'agrégation}
L'application des opérateurs d'agrégation a un surcoût.
%
Il est nécessaire de savoir à partir de quel moment le surcoût d'agrégation devient plus petit que le temps gagné par l'agrégation.
%
Dans notre cas l'agrégation est faite au début du programme et reste valide tant que la structure du réservoir ne change pas.
%
Le temps passé à appliquer les opérateurs d'agrégation va dépendre du nombre de tâche, des opérateurs et de l'ordre des opérateurs.
%
Par exemple pour la matrice SPE10 il faut 0,88~s pour appliquer l'opérateur C, 1,81~s pour appliquer l'opérateur D(2) et 1,24s pour appliquer CD(2).
%
Toujours pour le cas SPE10, il devient rentable d'utiliser Taggre à partir d'environ 7 factorisations ou 7 résolutions triangulaire.
%
Les 7 résolutions triangulaire peuvent être faite dans une résolution du GMRES.
