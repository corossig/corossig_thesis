\subsection{Processeurs mono-coeur}
Pour être capable de simuler de grands problèmes physiques, nous avons besoin de beaucoup de puissance de calcul.
%
Cette puissance se mesure en FLOPS\footnote{FLoating-point Operations Per Second}, il s'agit du nombre d'opérations par seconde qu'un ordinateur peut effectuer sur des nombres à virgules flottantes.
%
Même les processeurs mono-coeur peuvent faire des opérations en parallèle.
%
Le parallélisme au niveau instruction en est un bon exemple, les pipelines d'instructions permettent de paralléliser les différentes étapes liées au traitement d'une exécution.
%
Dans l'idéal, le processeur utilisant un pipeline d'instruction pourra exécuter une opération par cycle, donc un processeur à 4~GHz aura une puissance de calcul de 4~GFLOPS si les opérations flottantes sont faites en 1 cycle.


Par la suite, les processeurs ont gagnés des instructions permettant d'effectuer une même opération sur des données différentes, aussi appelées instructions SIMD dans la taxonomie de Flynn.
%
Ces processeurs dits vectoriel peuvent donc avoir une puissance de calcul supérieur, si une instruction est capable d'effectuer 4 opérations à la fois et qu'il tourne à 4~GHz, alors il aura une puissance de calcul de 16~GFLOPS.
%
Il s'agit ici d'une puissance théorique, tous les codes de calculs n'ont pas la possibilité d'exploiter les instructions vectorielles.
%
Ces instructions sont souvent utilisées dans les noyaux de calculs d'algèbre linéaire dense.
%
En simulation de réservoir, nous utilisons ces noyaux de calculs sur les blocs de nos matrices.
