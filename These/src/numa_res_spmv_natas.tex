\subsubsection{NATaS}
NATaS distribue les pages mémoires utilisées par la matrice en mode bloc de lignes.
%
Nous divisons le nombre de lignes par le nombre de bancs NUMA pour obtenir la taille d'un bloc.
%
Puis chaque bloc est alloué sur un banc NUMA différent.
%
En plus d'optimiser le placement des pages, NATaS répartira la charge de travail en prenant en compte la localité mémoire.
%
Sur le cas à 8 variables primaires, nous obtenons les mêmes performances qu'en mémoire distribuée (Fig.~\ref{fig:res_spmv_nas_rostand}).
%
Pour 3 variables primaires, nous obtenons 95~\% des performances de la version en mémoire distribuée.
%
Par contre, nous n'obtenons que 70~\% avec 1 variable primaire.




Malheureusement, sur la machine Manumanu, malgré un placement optimal des pages mémoires, NATaS ne fait pas aussi bien que MPI (Fig.~\ref{fig:res_spmv_nas_manu}).
%
Les compteurs matérielles nous indiquent qu'un nombre non négligeable d'accès à une mémoire distante sont faits.
%
Or, nous avons vérifié l'emplacement physique des pages mémoires de la matrices ainsi que celles des vecteurs et celles-ci sont bien placées.
%
Les accès à la mémoire distante proviennent donc de l'ordonnanceur de tâches.
%
C'est en grande partie de l'architecture logicielle de NATaS qui en sont la cause.
%
Nous effectuons trop d'accès à un ensemble de variables partagées par tous les threads.
%
Par exemple, nous avons un compteur de tâches en cours qui nous permet de savoir s'il existe encore des tâches pouvant être volées ou débloquées.
%
Le vol de travail est aussi consommateur d'accès mémoire distant.
%
Actuellement, lorsqu'un thread tombe à court de tâche, il parcourt les structures servant à maintenir la liste des tâches disponibles sur les autres bancs NUMA.
%
Pour améliorer considérablement les performances, il faudrait construire un ordonnanceur utilisant moins de variables partagées.
%
Mais ce type d'ordonnanceur est dur à écrire, nos machines de calculs étant essentiellement composées de 2 bancs NUMA, il n'y a aucune volonté de notre côté pour écrire ce type d'ordonnanceur.
%
L'utilisation de 2 bancs NUMA sur Manumanu nous donne une accélération de 10, tout comme la version en mémoire distribuée.
%
Au delà de 2 bancs NUMA, les performances sont légèrement meilleures que la version OpenMP, atteignant une accélération de 32.
