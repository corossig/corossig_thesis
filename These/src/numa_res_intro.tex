Nous allons maintenant présenter les résultats obtenues sur trois des noyaux de calcul du GMRES.
%
Dans un premier temps nous présenterons les résultats du produit matrice vecteur creux.
%
Puis nous présenterons les performances de la factorisation ILU ainsi que les résolutions triangulaires associées.
%
Pour évaluer les gains de notre solution, nous allons à chaque fois présenter les résultats obtenues avec les différentes stratégies d'allocations disponibles.
%
La stratégie consistant à utiliser la mémoire distribuée nous indiqueras une approximation de la performance maximale que nous pouvons atteindre en mémoire partagée.
%
Puis nous évaluerons les politiques d'allocation first touch et interleave.
%
Nous pourrons ensuite comparer ces résultats à la solution que nous avons développée : NATaS.
%
Et en dernier, nous testerons la politique d'équilibrage NUMA automatique mis en place dans les versions récentes du noyau Linux.
%
Les différents tests seront effectués à la fois sur Rostand et sur Manumanu, seul le test la politique d'équilibrage NUMA automatique sera effectué sur une machine différente.


Nous utiliserons trois cas tests qui correspondent tous à une matrice représentant un cube de 100 éléments de côté.
%
Nous faisons seulement varier le nombre de variables primaires pour faire varier la taille en mémoire du cas test ainsi que la possibilité de réutiliser des données en cache.
%
La variation de la taille du cube n'aurait permis que de faire varier la taille des cas tests.
%
Nous utiliserons 1, 3 et 8 variables primaires qui nous donnerons respectivement trois matrices dont les tailles mémoire seront 52~Mo, 470~Mo et 3,33~Go.
