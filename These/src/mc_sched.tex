\section{Politiques d'ordonnancement}
La politique d'ordonnancement aura un impact conséquent sur les performances d'un code.
%
Chaque politique aura un surcoût différent en fonction de sa complexité algorithmique ainsi que des paramètres qu'il prend en compte.

Un ordonnanceur type tourniquet distribuera les tâches de manière uniforme sur les différents coeurs de calcul.
%
La complexité algorithmique est donc minimale, mais aucune métrique n'est prise en charge.
%
Nous aurons donc un ordonnancement rapide avec un surcoût très faible, mais de très mauvaise qualité.

Parmi les ordonnanceurs simples, nous pouvons aussi parler de la queue partagée.
%
Tous les threads partagent une queue contenant les tâches pouvant être exécutées.
%
L'équilibrage de charge se fera façon automatique, tant que la queue n'est pas vide il reste du travail à effectuer.
%
Ce type d'ordonnancement impose une implémentation de la queue {\em thread-safe}, c'est à dire qui garantit les opérations d'ajout et de suppression de tâches dans la queue même lorsque plusieurs threads y accèdent simultanément.
%
Cette implémentation aura aussi des conséquences sur le surcoût d'ordonnancement des tâches.
%
Une implémentation à verrou aura un plus gros surcoût qu'une implémentation à base d'instructions atomiques.
%
Ce type d'ordonnancement ne garantit pas un bon équilibrage de charge.
%
Plusieurs cas peuvent conduire à un déséquilibre de charge :
\begin{itemize}
  \item si les dernières tâches à ordonnancer ont des coûts très différents;
  \item si une tâche qui débloque beaucoup de parallélisme est exécutée très tard.
\end{itemize}

L'algorithme d'ordonnancement HEFT est un algorithme performant et à faible complexité\cite{heft2}.
%
Il s'agit d'un algorithme glouton, il va essayer de toujours occuper tous les coeurs de calcul.
%
Le graphe de tâches est parcouru en hauteur et les tâches sont distribuées les unes à la suite des autres.
%
Pour chaque tâche un temps de terminaison est estimé pour chaque thread, le thread qui aura le temps de terminaison le plus court sera en charge d'exécuter la tâche.
%
Il existe des cas où cet algorithme n'effectue pas le meilleur ordonnancement possible.
%
Comme par exemple lorsqu'une tâche qui en débloque beaucoup d'autres est exécutée en dernière.
