\section{Politiques d'ordonnancement}
La politique d'ordonnancement aura un impact conséquent sur les performances d'un code.
%
Chaque politique aura un surcoût différent en fonction de sa complexité algorithmique ainsi que des paramètres qu'il prend en compte.


Un ordonnanceur type tourniquet distribueras les tâches de manière uniforme sur les différents coeurs de calcul.
%
La complexité algorithmique est donc minimal, mais aucune métrique n'est pris en charge.
%
Nous aurons donc un ordonnancement rapide avec un surcoût faible mais de très mauvaise qualité.


Parmi les ordonnanceurs très simple, nous pouvons aussi parler de la queue partagée.
%
Tous les threads partagent une queue contenant les tâches pouvant être exécutées.
%
L'équilibrage sera de meilleur qualité
%
Ce type d'ordonnancement impose une implémentation de la queue thread-safe
