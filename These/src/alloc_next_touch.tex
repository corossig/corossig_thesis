\subsection{Next touch}
\label{sec:next_touch}
L'idée du First touch n'est pas mauvaise, mais elle impose une phase d'initialisation parallèle.
%
Au lieu de récrire toute l'initialisation d'un programme, il pourrait être intéressant d'utiliser les phases de calculs pour distribuer la mémoire sur tous les bancs NUMA.
%
C'est pour cela que la politique d'allocation {\em Next touch} a été créée.
%
Le programmeur choisit un ensemble de pages mémoires qu'il pense mal placées et définit une politique d'allocation next touch sur ces pages.
%
Lors du prochain accès mémoire à l'une de ces pages, le noyau s'occupera, si besoin, de déplacer la page mémoire vers le banc NUMA le plus proche du processeur faisant cet accès.
%
Ainsi nous pouvons obtenir une amélioration de la localité mémoire sans avoir à récrire certaines parties du code.
%
Cette politique aurait pu apporter des performances supplémentaires à notre code, mais n'étant pas disponible dans le noyau Linux, malgré des propositions d'extensions~\cite{next_touch_linux,GoFu09Next-touch}, nous n'avons pas pu l'utiliser telle quelle.
%
\`A la place, nous avons implémenté une solution similaire qui consiste à choisir manuellement l'emplacement des pages mémoires.
