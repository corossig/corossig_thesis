\subsubsection{Généralisé}
Les opérateurs F et C s'occupent chacun d'un des problèmes que l'on souhaite résoudre avec Taggre.
%
Mais nous pouvons essayer de généraliser leurs propriétés dans un seul opérateur.
%
Ainsi, nous avons construit l'opérateur G.
%
Son but est de privilégier une direction d'agrégation qui puisse être une combinaison de la hauteur et de la largeur.
%
Si seule la largeur est privilégiée, nous nous rapprochons de l'opérateur F.
%
Mais ces deux algorithmes ne sont pas équivalents, la taille des tâches de l'opérateur F n'est pas fixe, elle dépend du nombre de tâches à une hauteur donnée alors que l'opérateur G crée des tâches de taille fixe.
%
En privilégiant les tâches de type continuation, nous retrouvons l'opérateur C.
%
Cette sélection se fait à l'aide de la fonction de tri dans l'algorithme~\ref{algo:algo_G}.
%
Malheureusement cette fonction de tri n'est pas facilement définissable, elle dépendra du problème à résoudre.
%
Nous n'avons donc pas implémenté directement cet algorithme mais nous avons préféré créer un opérateur qui définit une fonction de tri.


\begin{algorithm}
  \caption{Algorithme de l'opérateur généralisé.}
  \label{algo:algo_G}
  \KwData{DAG, $M$ : le nombre de tâches dans un agrégat}
  {\sc Prêt} = liste vide \\
  mettre les tâches racines de DAG dans {\sc Prêt} \\
  \While{{\sc Prêt} n'est pas vide} {
    {\sc Profondeur} = {\sc Prêt} \\
    {\sc Prêt} = liste vide \\
    \While{{\sc Profondeur} n'est pas vide} {
      {\sc Maître} = retirer le premier de {\sc Profondeur} \\
      {\sc Libérer} = liste vide \\
      mettre {\sc Maître} dans {\sc Libérer} \\
      {\sc Compteur} = 0 \\
      \While{{\sc Compteur} $< M$ {\bf et} {\sc Libérer} n'est pas vide} {
        {\sc Suivant} = retirer le premier de {\sc Libérer} \\
        {\sc Compteur}++\\
        {\sc Maître} devient {\sc Maître} union {\sc Suivant}\\
        mettre les tâches libérées par {\sc Suivant} dans {\sc Libérer} \\
        appliquer une fonction de tri sur {\sc Libérer} \\
      }
      mettre les tâches libérées par {\sc Maître} dans {\sc Profondeur} \\
    }
    mettre les tâches libérées par {\sc Profondeur} dans {\sc Prêt}\\
  }
\end{algorithm}
