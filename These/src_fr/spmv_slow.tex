%+++++++++++++++++++++++++++++++
\section{Pourquoi mon SpMV est si lent ?}
%-------------------------------
\subsection{Deux fois plus de coeurs mais pas deux fois plus rapide}
  \begin{itemize}
    \item Just draw some curves
  \end{itemize}
%-------------------------------
\subsection{Explication de la mémoire}
Pour comprendre les résultats précédent nous devons étudier un petit peu l'architecture des ordinateurs.
%
Les performances d'un ordinateur dépendent essentiellement de deux choses : le processeur et la mémoire.
%
Quand un programme souhaite lire une donnée en mémoire, il subit une pénalité mémoire.
%
Cette pénalité est la somme de deux contraintes :
\begin{itemize}
        \item la latence : la différence de temps entre la demande d'accès à la mémoire et la réception du premier octet;
        \item la bande passante : le nombre maximum d'octet par seconde que le bus peut envoyer/recevoir
\end{itemize}
%
In a modern computer, there is different type of memories.
%




  \begin{itemize}
    \item Define memory penalty
    \item Talk about bandwidth
  \end{itemize}
%+++++++++++++++++++++++++++++++
