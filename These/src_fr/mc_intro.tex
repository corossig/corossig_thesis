Lorsque l'on souhaite paralléliser un code de calcul, on se retrouve à devoir choisir parmi plusieurs paradigme de parallélisation.
%
Le choix de ce paradigme est une étape importante, elle déterminera les algorithmes a utilisés et donc aussi les performances du programme.
%
En effet, un même problème ne se résoudra pas de la même façon en fonction du paradigme choisi.
%
Mais le choix du paradigme est aussi déterminé par l'architecture de la machine cible.
%
Dans le cas d'une grappe de serveurs, on préféra un paradigme par passage de messages qui nous obligera à utiliser des algorithmes distribués.
%
Alors que dans le cas d'une machine à mémoire partagée nous aurons recours à l'utilisation de processus légers, aussi appelé {\em thread}.
%
Plusieurs paradigmes peuvent être utilisés ensemble, nous pouvons ainsi tirer parti des avantages de chacun tout en limitant leurs inconvénients.
%
Nous allons maintenant détailler les différents paradigmes de parallélisation que nous avons utilisé.
