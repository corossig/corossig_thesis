\subsection{GMRES préconditionné}
Une approche souvent utilisée pour résoudre de grands systèmes d'équations linéaires creux consiste à utiliser des méthodes de résolution itérative.
%
Cette partie représente souvent la partie qui consomme le plus de temps dans une simulation numérique, par exemple dans la simulation de réservoir cela peut représenter jusqu'à 80~\% du temps de simulation.
%
On appel méthode itérative une méthode qui permet de résoudre un problème en partant d'une solution initial $x^0$ et qui à chaque itération donne une nouvelle solution $x^i$.
%
Cette nouvelle solution $x^i$ étant plus proche de la solution exacte du problème que la solution précédente $x^{i-1}$.
%
La méthode s'arrête lorsque $x^i$ est suffisamment proche de la solution exacte selon un critère entré en paramètre.
%
Parmi ces méthodes on peut citer la méthode Jacobi, Gauss-Seidel ou encore SOR, ce sont des méthodes itératives dites stationnaires.
%
Mais ces méthodes ne sont pas génériques, leur convergence dépend de certaines propriétés de la matrice.
%
Utilisées tel quel, ces méthodes ne convergent pas rapidement dans de nombreux cas concret.


La méthode du gradient conjugué est une méthode qui s'applique seulement à des matrices carrées symétriques définies positives.
%
Cette méthode permet de converger en au plus $n$ itérations avec $n$ la dimension de la matrice.
%
Mais avec un bon préconditionnement, on obtient rapidement une solution très proche de la solution exacte.
%
Puis cette méthode a été étendue aux matrices non-symétriques sous le nom du gradient biconjugué.
%
Le gradient biconjugué est une méthode par projection dans un espace de Krylov.
%
Le GMRES est aussi une méthode de Krylov, elle fonctionne avec n'importe quelle matrice du moment que celle-ci soit inversible (voir~\cite{Saad96IMSLS}).
%
L'algorithme du GMRES est composé d'opérations sur des vecteurs ainsi que d'un SpMV\footnote{produit matrice-vecteur creux}.

Comme les matrices utilisées dans la simulation de réservoir ne sont pas bien conditionnées, l'algorithme du GMRES converge après beaucoup d'itérations.
%
Dans ce cas, nous devons préconditionner la matrice pour faire en sorte que le GMRES converge avec moins d'itérations.
%
Il faut choisir une matrice $M^{-1}$ tel que $M^{-1}A$ soit mieux conditionnées que $A$.
%
Un cas idéal serait d'avoir $M=A$, dans ce cas là on obtient la matrice identité qui se trouve très bien conditionnée.
%
Or, calculer $A^{-1}$ est très coûteux, à la fois en terme de calcul que de mémoire.

La factorisation ILU\footnote{Incomplete LU} est un bon préconditionneur pour nos matrices car il se rapproche de $A^{-1}$.
%
Cette méthode est composée de deux opérations, la première correspond à la {\em factorisation} de la matrice en deux sous matrices et la deuxième correspond à la {\em résolution triangulaire} effectuée avec les deux sous matrices.
%
La factorisation LU correspond à la factorisation d'une matrice $A$ en deux matrices triangulaires $L$ et $U$.
%
Puisque résoudre l'équation $Ax=b$ est équivalent à résoudre $Ly=b$ et $U.x=y$, ces deux résolutions peuvent être faites rapidement parce que les matrices $L$ et $U$ sont triangulaires.
%
Dans le cas de problèmes linéaires creux, le résultat de la factorisation exacte de la matrice creuse $A$ ne pourrait plus être considéré comme creux.
%
En effet, beaucoup de valeurs nulles deviendraient non nulles et l'espace mémoire nécessaire au stockage de ces valeurs deviendrait gigantesque.
%
Pour maintenir un espace mémoire raisonnable, on peut faire seulement une partie de la factorisation et considérer le reste comme négligeable, il s'agit de la factorisation incomplète.
%
Dans l'algorithme ILU, on essaie d'obtenir un motif creux pour $L$ et $U$ aussi proche que possible du motif creux de $A$.
%
Il y a deux façons équivalentes pour appliquer l'algorithme ILU dans GMRES :
\begin{itemize}
  \item le préconditionnement à gauche : $M^{-1}(Ax)=b$;
  \item le préconditionnement à droite : $A(M^{-1}x)=b$
\end{itemize}
%
%In programming term, this means that the SpMV must be done before or after the TRSV.
