\subsection{Cas d'étude}
Pour être en mesure de tester notre méthode de parallélisation en mémoire partagée, nous utilisons un code de solveur linéaire développé à Total SA.
%
Nous allons essayer de paralléliser la partie RAS~\footnote{Restrictive Additive Schwartz} du code.
%
Dans le but d'évaluer le gain de performance, nous avons choisi des systèmes linéaires à résoudre avec le solveur linéaire.
%
Nous allons utiliser le cas test SPE10, ce cas est basé sur les données prises du second modèle du 10ème cas test SPE\cite{SPE10}.
%
C'est un réservoir de 1~122~000 de cellules, organisées dans une grille 3D cartésienne de taille 60 x 220 x 85, il s'agit de schéma de discrétisation en 7 points et c'est un problème de référence dans l'industrie du pétrole.
%
Les autres cas tests seront générés par un programme développé en interne, et il s'agit aussi d'un schéma de discrétisation en 7 points (e.g., volume fini).
%
Il génère des cubes 3D cartésiens de taille arbitraire.
%
Ces cas générés nous permettent de tester énormément de combinaisons de tailles dans le but d'évaluer le passage à l'échelle de nos algorithme.
