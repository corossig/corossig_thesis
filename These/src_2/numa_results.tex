%-------------------------------
\section{Result}
%On a single 4-core socket, the Nas scheduler delivers results very similar to those
%of TBB because no NUMA is involved. On our NUMA platform with two 4-core sockets,
%the Nas scheduler delivers better results than TBB. On the
%factorization step, we achieve an improvement
%from 1 to 40\,\% (Tables~\ref{tab:full:8:facto:no},~\ref{tab:full:8:facto:nested}).

%On the triangular solve step, the
%improvement obtained with the Nas scheduler is 12.36\,\% on average
%(Tables~\ref{tab:full:8:solve:no},~\ref{tab:full:8:solve:nested}).
%% On Cube\_100 case, interleaved method is more effective than Nas method because of memory access
%% to the vector. For now, we allocate the vector with the first NUMA helper of Nas,
%% but during triangular solve, the first part of the vector is only write by first NUMA node, then
%% gradually the second NUMA node interleave some write in the middle part of the vector and finally only the second NUMA node
%% write the end of the vector. We are working on a generic way to optimize memory distribution.

%On Cube\_100 case, one can notice that the interleaved memory allocation method
%gives better results than using our NUMA aware allocator. This is due to the fact
%that in the triangular solve, a task accesses the matrix data and some part of
%the vector. The problem is that the matrix is allocated using the second NUMA helper
%allocator (described ini Subsection \ref{NUMA_helper}) whereas the vector is
%allocated using the first NUMA helper allocator (optimized for the matrix-vector
%product). Hence, the memory accesses to the vector are not optimized in the
%triangular solve and the interleaved memory allocation method gives better results
%in average. One can see that in the case SPE10 where each entry of the matrix
%is a dense block (3,3), the memory accesses to the vector are more neglectable
%and in this case the NUMA allocator is better.


%-------------------------------
\subsection{First touch}
%-------------------------------
\subsection{Interleave}
%-------------------------------
\subsection{Automatic NUMA balancing}
%-------------------------------
\subsection{NAS}

(TODO: faire des tableaux pour comparer une execution avec OpenMP (le meilleurs scheduler non NUMA que l'on possède), OpenMP avec le support autoNUMA de linux et Nas)
