Each fine grain task performs one elementary line operation. SPE10
tasks weight 2-5 times more than Cube\_100 tasks. However, both
cases generate approximately 1 million tasks each.

The test protocol is the following:
\begin{itemize}
\item Each test is run three times, the final retained result is
  computed as the average of the three measured results. For each
  test, we collect three different timings:
  factorization time, triangular solve time, and aggregation time.

\item We perform the tests on a single socket (using
      {\em numactl --cpunodebind}) and on two sockets.

  % OA: je pense qu'un mot d'explication sur l'ordering sera
  % necessaire

\item We test two orderings (set of row/column permutations): A {\em natural} ordering which corresponds to no
      modification on matrix structure where unknowns are sequentially
      ordered by plane along the geometric z axis
      (this leads to a perfect seven diagonals pattern for Cube\_100)
      and a {\em nested dissection}~\cite{Saad96IMSLS} ordering which exposes more parallelism.

  % ST: TODO: expliquer brièvement ce que sont les natural et nest
  % dissection orderings.
  % ST: TODO: expliquer pourquoi ça n'a pas de sens d'appliquer
  % D() au cas nested dissection.

\item We test three levels of ILU(k) fill: 0, 1 and 2.
      The level $k$ of ILU(k) preconditioner determines the level of fill of
      the matrix. In others words, computation per line and the number of dependencies
      between tasks grow up with a greater $k$. Table~\ref{tab:edges} gives the number of edges resulting in the DAG of the factorization
      (triangular solve DAG has a similar number of edges) for ILU$(0)$, ILU$(1)$ and ILU$(2)$. Subsection \ref{precond_step} will also give some features on the cost of a computational task (one row factorization)
      depending on the ILU level of fill parameter.

\item For the parameterized aggregation heuristics, we select the
      parameter values leading to the best performance result. With
      natural ordering, we can use the Cache Oriented algorithm because of
      particular DAG structure. With nested dissection ordering, the Cache Oriented algorithm
      can't be used, so we use the Depth Front algorithm with a high value.
\end{itemize}



%-------------------------------
\subsection{ILU(k) Factorization Step}
\label{precond_step}
The first test series is performed on the ILU(k) factorization step
on a single 4-core socket (Tables~\ref{tab:tbb:4:facto:no},~\ref{tab:tbb:4:facto:nested}).

% ST: TODO: mentionner que ça consiste ici en trois phases de
% préconditionnement: ILU(0), ILU(1), ILU(2), et indiquer brièvement la
% différence entre ces phases. Tout le monde ne connait pas
% GMRES, loin de là:)
With aggregation disabled the task-parallel ILU(0) factorization is always
slower than the sequential version. This is due to the additional cost
of task management. Tasks duration for CUBE\_100 in ILU(0) is only $50\,ns$ and
$240\,ns$ for SPE10, but one task
management duration is approximately $500\,ns$. In ILU(2), tasks are bigger, their
duration is $600\,ns$ for CUBE\_100 and $1.7\,\mu{s}$ for SPE10 but it's
not bigger enough to consider task management negligible.
Another important aspect of the ILU factorization and triangular solve is that on our testbed machine
the algorithm speed-up is bounded by the memory bandwidth: Therefore the maximum theoretical speed-up achievable
by such algorithms on several cores is less than the number of cores used.

With aggregation enabled and {\em CD(4)} coarse strategy string, we now reduce the number
of tasks to 2,500 with a task duration 400 times bigger.
In ILU(0) we achieve a moderate speed-up of 2. ILU(2) factorization achieves a
better speed-up of 3. The Front algorithm is not as effective as
the Depth Front in this test case, because it doesn't aggregate tasks
with continuous lines, which cause many cache misses.

% ST: TODO: Que veut dire contiguous data?
% ST: TODO: est-ce que ça aide de faire plusieurs D(4) ? En faire
% plus aurait-il aidé plus ? Discuter de quand s'arrêter ?

With the Front algorithm ({\em F(32)}) in ILU(0) with natural ordering,
%tasks at the begin and the end of the DAGs are not aggregated,
we aggregate a maximum of 157 tasks together and, on average, we aggregate only 53 tasks.



\begin{table}[!h]
  \renewcommand{\arraystretch}{1.3}
  \caption{Results on the ILU(k) factorization step on a single 4-core
    socket with TBB with nested dissection ordering.}
  \label{tab:tbb:4:facto:nested}
  \centering
  \begin{tabular}{|c|c||c|c|c|c|}
    \hline
    Matrix & ILU & Sequential & No agg. & F(32) & D(400)\\
    &     &  (second)  & \multicolumn{3}{c|}{(speed-up)}\\
    \hline
    \hline
    & ILU(0) & 0.129 & 0.46 & 2.17 & 2.26\\
    CUBE\_100 & ILU(1) & 0.495 & 1.29 & 1.70 & 2.78\\
    & ILU(2) & 0.828 & 1.74 & 1.94 & {\bf 3.12}\\
    \hline
    & ILU(0) & 0.276 & 0.74 & 1.93 & 2.16\\
    SPE10     & ILU(1) & 1.375 & 1.98 & 1.67 & 2.82\\
    & ILU(2) & 2.247 & 2.43 & 1.85 & {\bf 3.12}\\
    \hline
  \end{tabular}
\end{table}

With two 4-core sockets (Tables~\ref{tab:tbb:8:facto:no},~\ref{tab:tbb:8:facto:nested}), the parallel
ILU(0) again performs slower than sequential execution when
aggregation is disabled. With aggregation enabled, the ILU(0)
achieves a speed-up of 3. ILU(2) achieves a speed-up of 6.2.

\begin{table}[!h]
  \renewcommand{\arraystretch}{1.3}
  \caption{Results on the ILU(k) factorization step on two 4-core
    sockets with TBB with natural ordering.}
  \label{tab:tbb:8:facto:no}
  \centering
  \begin{tabular}{|c|c||c|c|c|c|}
    \hline
    Matrix & ILU & Sequential & No agg. & F(32) & CD(4)\\
    &     &  (second)  & \multicolumn{3}{c|}{(speed-up)}\\
    \hline
    \hline
    \hline
    & ILU(0) & 0.056 & 0.28 & 1.27 & 2.54\\
    CUBE\_100 & ILU(1) & 0.143 & 0.65 & 1.89 & 3.79\\
    & ILU(2) & 0.612 & 1.47 & 3.06 & 3.91\\
    \hline
    & ILU(0) & 0.260 & 0.97 & 2.48 & 3.78\\
    SPE10     & ILU(1) & 0.771 & 2.24 & 3.80 & 5.72\\
    & ILU(2) & 2.006 & 3.37 & 3.97 & {\bf 6.21}\\
    \hline
  \end{tabular}
\end{table}


\begin{table}[!h]
  \renewcommand{\arraystretch}{1.3}
  \caption{Results on the ILU(k) factorization step on two 4-core
    sockets with TBB with nested dissection ordering.}
  \label{tab:tbb:8:facto:nested}
  \centering
  \begin{tabular}{|c|c||c|c|c|c|}
    \hline
    Matrix & ILU & Sequential & No agg. & F(32) & D(400)\\
    &     &  (second)  & \multicolumn{3}{c|}{(speed-up)}\\
    \hline
    \hline
    & ILU(0) & 0.127 & 0.41 & 3.10 & 3.31\\
    CUBE\_100 & ILU(1) & 0.483 & 1.31 & 2.70 & 4.77\\
    & ILU(2) & 0.817 & 1.96 & 3.18 & 5.46\\
    \hline
    & ILU(0) & 0.277 & 0.71 & 2.59 & 3.09\\
    SPE10     & ILU(1) & 1.452 & 2.48 & 2.81 & 5.00\\
    & ILU(2) & 2.347 & 3.29 & 3.12 & {\bf 5.57}\\
    \hline
  \end{tabular}
\end{table}

%-------------------------------
\subsection{Triangular Solve Step}\label{subsec:solve}

The triangular solve step is itself composed of two parts: A
{\em forward} substitution followed by a {\em backward}
substitution. The DAG of the forward substitution is identical to
the DAG of the factorization step mentioned in the previous section.
Thus we can reuse the same coarse
DAG. In our test cases, the DAG of the backward substitution part is
the transpose of the DAG of the forward substitution. Here again,
the factorization coarse DAG can thus straightforwardly be reused.
In total we have twice more tasks in triangular solve than in factorization.

The weight of operations done in triangular solve elementary tasks
is lighter than their factorization task counterparts.
%% and these tasks need to access to more data..
Tests on a
single 4-core socket (Tables~\ref{tab:tbb:4:solve:no},~\ref{tab:tbb:4:solve:nested})
show that the parallel triangular solve is always slower than
the sequential version with task aggregation disabled. With
aggregation enabled, we obtain a speedup of~2.

% ST: TODO: idéalement il faudrait que ces deux tables se retrouvent
% côte à côte dans la version finale.
\begin{table}[!h]
  \renewcommand{\arraystretch}{1.3}
  \caption{Results on the Triangular Solve step on a single 4-core
    socket with TBB with natural ordering.}
  \label{tab:tbb:4:solve:no}
  \centering
  \begin{tabular}{|c|c||c|c|c|c|}
    \hline
    Matrix & ILU & Sequential & No agg. & F(32) & CD(4)\\
    &     &  (second)  & \multicolumn{3}{c|}{(speed-up)}\\
    \hline
    \hline
    & ILU(0) & 0.092 & 0.23 & 0.88 & 1.90\\
    CUBE\_100 & ILU(1) & 0.117 & 0.27 & 0.82 & 1.97\\
    & ILU(2) & 0.163 & 0.34 & 0.92 & 2.05\\
    \hline
    & ILU(0) & 0.219 & 0.40 & 1.08 & 1.93\\
    SPE10     & ILU(1) & 0.353 & 0.62 & 1.37 & 2.22\\
    & ILU(2) & 0.554 & 0.85 & 1.58 & {\bf 2.39}\\
    \hline
  \end{tabular}
\end{table}

\begin{table}[!h]
  \renewcommand{\arraystretch}{1.3}
  \caption{Results of the Triangular Solve step on a single 4-core
    socket with TBB with nested dissection ordering.}
  \label{tab:tbb:4:solve:nested}
  \centering
  \begin{tabular}{|c|c||c|c|c|c|}
    \hline
    Matrix & ILU & Sequential & No agg. & F(32) & D(400)\\
    &     &  (second)  & \multicolumn{3}{c|}{(speed-up)}\\
    \hline
    \hline
    & ILU(0) & 0.107 & 0.20 & 1.19 & 1.34\\
    CUBE\_100 & ILU(1) & 0.150 & 0.31 & 0.94 & 1.39\\
    & ILU(2) & 0.171 & 0.34 & 0.95 & 1.50\\
    \hline
    & ILU(0) & 0.249 & 0.37 & 1.39 & 1.52\\
    SPE10     & ILU(1) & 0.430 & 0.65 & 1.32 & 1.77\\
    & ILU(2) & 0.500 & 0.72 & 1.35 & {\bf 1.89}\\
    \hline
  \end{tabular}
\end{table}

On two 4-core sockets
(Tables~\ref{tab:tbb:8:solve:no},~\ref{tab:tbb:8:solve:nested}), with
task aggregation disabled, only the SPE10 test achieves a speedup
greater than 1. With aggregation enabled, a speedup of 2.52 is
achieved on ILU(0) and a speedup of 4.14 is achieved on ILU(2).

\begin{table}[!h]
  \renewcommand{\arraystretch}{1.3}
  \caption{Results of the Triangular Solve step on two 4-core sockets with
    TBB with natural ordering.}
  \label{tab:tbb:8:solve:no}
  \centering
  \begin{tabular}{|c|c||c|c|c|c|}
    \hline
    Matrix & ILU & Sequential & No agg. & F(32) & CD(4)\\
    &     &  (second)  & \multicolumn{3}{c|}{(speed-up)}\\
    \hline
    \hline
    & ILU(0) & 0.092 & 0.27 & 1.27 & 2.44\\
    CUBE\_100 & ILU(1) & 0.123 & 0.35 & 1.54 & 2.82\\
    & ILU(2) & 0.174 & 0.45 & 1.40 & 2.98\\
    \hline
    & ILU(0) & 0.219 & 0.53 & 1.63 & {\bf 2.52}\\
    SPE10     & ILU(1) & 0.408 & 0.96 & 2.39 & 3.77\\
    & ILU(2) & 0.658 & 1.38 & 2.79 & {\bf 4.14}\\
    \hline
  \end{tabular}
\end{table}


\begin{table}[!h]
  \renewcommand{\arraystretch}{1.3}
  \caption{Results of the Triangular Solve step on two 4-core
    sockets with TBB with nested dissection ordering.}
  \label{tab:tbb:8:solve:nested}
  \centering
  \begin{tabular}{|c|c||c|c|c|c|}
    \hline
    Matrix & ILU & Sequential & No agg. & F(32) & D(400)\\
    &     &  (second)  & \multicolumn{3}{c|}{(speed-up)}\\
    \hline
    \hline
    & ILU(0) & 0.107 & 0.18 & 1.41 & 1.67\\
    CUBE\_100 & ILU(1) & 0.156 & 0.32 & 1.36 & 1.92\\
    & ILU(2) & 0.180 & 0.36 & 1.37 & 2.08\\
    \hline
    & ILU(0) & 0.249 & 0.35 & 1.64 & 1.82\\
    SPE10     & ILU(1) & 0.496 & 0.80 & 2.12 & 2.71\\
    & ILU(2) & 0.578 & 0.90 & 2.19 & {\bf 2.90}\\
    \hline
  \end{tabular}
\end{table}

% In most cases, domain decomposition achieves better raw performance
% results. However, by avoiding potentially harmful domain boundaries artifacts, the
% aggregation method ensures a better overall numerical stability than the
% domain decomposition.
